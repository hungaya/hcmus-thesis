\documentclass[12pt,a4paper]{article}
\usepackage[utf8]{vietnam}
\usepackage[left=2.5cm, right=3cm, top=2.00cm, bottom=2.00cm]{geometry}
\usepackage{fancyhdr}
\pagestyle{fancy}
\fancyhf{}
%\fancyhead[R]{\textit {Đỗ Sỹ Hưng}}
%\fancyhead[L]{Luận văn }
\fancyfoot[L]{\nouppercase{\leftmark}}
\fancyfoot[R]{\thepage}
\rfoot{\thepage}



\usepackage{chngcntr}
\counterwithin{equation}{section} 
\usepackage{tikz}
\usetikzlibrary{snakes}
\usepackage{rotating}
\usepackage{amsmath}
\usepackage{mathtools}
\usepackage{longfbox}
\usepackage{chronology}
\usepackage{graphicx}
\usepackage{amssymb} % $\mathbb{R}
\usepackage{amsthm} % Unnumbered theorem
\usepackage{graphicx} %insert picture
\usepackage{tabu}
\usepackage[define-L-C-R]{nicematrix}
\usepackage{enumitem} %a, b, c [label=(\alph*)]; [label=(\Alph*)]; [label=(\roman*)] [label=(\arabic*)]
\graphicspath{ {images/} } %insert picture
% \renewcommand\refname{A}
\def\N{\mathbb{N}}
\def\Z{\mathbb{Z}}
\def\Q{\mathbb{Q}}
\def\R{\mathbb{R}}
\NiceMatrixOptions{cell-space-top-limit=1pt,cell-space-bottom-limit=1pt}
\newcommand{\Char}{{\mathrm char}}
\newcommand{\im}{{\mathrm im}}
\newcommand{\setheight}[2]{\smash{#1}\vphantom{#2}}
\DeclareMathOperator{\esssup}{esssup}
\DeclareMathOperator{\supp}{supp}
\newtheorem{theorem}{Bài toán}[section]
\newtheorem{prop}[theorem]{}
\newtheorem{proposition}[theorem]{Mệnh đề}
\newtheorem{lemma}[theorem]{Bổ đề}
\newtheorem{corollary}[theorem]{Corollary}
% \theoremstyle{definition}
\newtheorem{definition}[theorem]{Định nghĩa}
\newtheorem{remark}[theorem]{Nhận xét}
\newtheorem{warning}[theorem]{Warning}
\theoremstyle{definition}
\newtheorem{rem}{Remark}[subsection]
\newtheorem{example}{Example}[subsection]
\newenvironment{solution}
  {\renewcommand\qedsymbol{$\blacksquare$}\begin{proof}[ minh]}
  {\end{proof}}
\pagestyle{fancy}

% for the abstract
% Abstracts have to be 12pt, indented 1.4 inches on each side, and inline with the label
\renewenvironment{abstract}{%
\noindent\begin{minipage}{1\textwidth}
\setlength{\leftskip}{0.4in}
\setlength{\rightskip}{0.4in}
\textbf{Abstract.}}
{\end{minipage}}

\lhead{\it Phụ lục khoá luận}

\rhead{\it Ngành: Toán Giải tích}


\renewcommand{\headrulewidth}{1,2pt}

\renewcommand{\footrulewidth}{1,2pt} % Cái này là tiêu đề chạy
\begin{document}
\begin{titlepage}

\thispagestyle{empty}

\begin{center}
\vspace*{0.3cm}
{\fontsize{14}{1}\selectfont  ĐẠI HỌC QUỐC GIA TP. HỒ CHÍ MINH}\\
{\fontsize{14}{1}\selectfont\bf TRƯỜNG ĐẠI HỌC KHOA HỌC TỰ NHIÊN}\\
\hfill

\vspace*{3cm}

{\fontsize{14}{1}\selectfont\bf ĐỖ SỸ HƯNG}

\vspace*{3cm}

{\fontsize{16}{1}\selectfont \bf { TÍNH CHẤT NGHIỆM CỦA BÀI TOÁN DIRICHLET CHO PHƯƠNG TRÌNH KIRCHHOFF PHI TUYẾN CHỨA SỐ HẠNG TẮT DẦN MẠNH }}

\vspace*{4cm}

{\fontsize{14}{1}\selectfont\bf PHỤ LỤC \\ KHOÁ LUẬN TỐT NGHIỆP ĐẠI HỌC}

\vfill

{\fontsize{12}{1}\selectfont Thành phố Hồ Chí Minh - 2023}
\end{center}
\vspace*{0.5cm}


\end{titlepage}

\newpage
\thispagestyle{empty}


\begin{center}
\vspace*{0.3cm}
{\fontsize{14}{1}\selectfont  ĐẠI HỌC QUỐC GIA TP. HỒ CHÍ MINH}\\
{\fontsize{14}{1}\selectfont\bf TRƯỜNG ĐẠI HỌC KHOA HỌC TỰ NHIÊN}\\
\vspace*{3cm}

{\fontsize{14}{1}\selectfont\bf ĐỖ SỸ HƯNG}

\vspace*{3cm}

{\fontsize{16}{1}\selectfont \bf {TÍNH CHẤT NGHIỆM CỦA BÀI TOÁN DIRICHLET CHO PHƯƠNG TRÌNH KIRCHHOFF PHI TUYẾN CHỨA SỐ HẠNG TẮT DẦN MẠNH
}}

\vspace*{2cm}

\end{center}
 


\begin{center}
    {\fontsize{13}{1}\selectfont \qquad Ngành: \textbf{Toán Giải tích}} 
\end{center}
 


\vspace*{2.5cm}

\begin{center}
    {\fontsize{13}{1}\selectfont \qquad Giáo viên hướng dẫn}

    {\fontsize{13}{1}\selectfont \qquad \textbf{TS. Nguyễn Thành Long}}
\end{center}
 


\vfill

\begin{center}
{\fontsize{12}{1}\selectfont  Thành phố Hồ Chí Minh - 2023}
\end{center}
 

\vspace*{0.5cm}

\newpage
\tableofcontents
\newpage

\section{Chương tổng quan}
\section{Kiến thức chuẩn bị}
\section{Sự tồn tại và xấp xỉ của nghiệm yếu}

\subsection{Giới thiệu}

\begin{theorem}
Bài toán
\begin{align} \label{maineqn}
    \begin{cases}
        u_{tt} - \sigma u_{txx} - \left(1 + \|u_x(t)\|^2\right) u_{xx} + Ku^3 + \lambda u_t = f(x,t), 0 < x < 1, 0 < t < T \\
        u(0,t) = u(1,t) = 0, \\
        u(x,0) = \tilde{u}_0(x),\ u_t(x,0) = \tilde{u}_1(x).
    \end{cases}
\end{align}
có nghiệm yếu $u$ thoả bài toán biến phân sau
\begin{align}
    \left< u''(t), v \right> + \sigma \left< u'_x(t), v_x \right> + &\left(1 + \|u_x(t)\|^2\right) \left< u_x(t), v_x \right> \\
    &+ K \left< u^3(t), v \right> + \lambda \left< u'(t), v \right> = \left< f(t), v \right>. \notag
\end{align}
\end{theorem}

\textit{Chứng minh.}

Ta biến đổi tích phân và sử dụng tích phân từng phần cho từng số hạng trong phương trình $\eqref{maineqn}_1$.

Đối với số hạng thứ nhất của vế trái
\begin{align*}
    \left<u_{tt}(t), v\right> = \left<u''(t), v\right>.
\end{align*}

Số hạng thứ hai của vế trái
\begin{align*}
    -\sigma \left<u_{txx}(t), v\right>
    &= -\sigma \left<u'_{xx}(t), v\right> \\
    &= -\sigma \left(\left.u'_x(x,t) v(x)\right|_{x=0}^{x=1} - \left<u'_x(t),v_x\right> \right)
    = \sigma \left<u'_x(t), v_x\right>.
\end{align*}

Số hạng thứ ba của vế trái
\begin{align*}
    -\left(1 + \|u_x(t)\|^2\right)\left<u_{xx}(t),v\right>
    &= -\left(1 + \|u_x(t)\|^2\right)\left(\left.u_x(x,t)v(x)\right|_{x=0}^{x=1} - \left<u_x(t),v_x\right>\right) \\
    &= \left(1 + \|u_x(t)\|^2\right) \left<u_x(t),v_x\right>.
\end{align*}

Các số hạng còn lại
\begin{align*}
    &K\left<u^3(t),v\right>, \\
    &\lambda \left<u_t(t),v\right> = \lambda \left<u'(t),v\right>, \\
    &\left<f(t),v\right>.
\end{align*} \qed

\subsection{Thuật giải xấp xỉ tuyến tính}
\subsection{Sự tồn tại của dãy lặp}

\begin{theorem}
    Cơ sở Hilbert $\{w_j\}$ của $L^2$, $w_j = \sqrt{2} \sin(j\pi x), j = 1,2,\dots$ được thành lập từ toán tử Laplace $-\Delta = \frac{-\partial^2}{\partial x^2}$ sao cho
    \begin{align*}
        -\Delta w_j = \lambda_j w_j, \: w_j \in H^1_0 \cap C^\infty([0,1]), \: \lambda_j = (j\pi)^2, j=1,2,\dots
    \end{align*}
    Hơn nữa, $\left\{\frac{w_j}{\sqrt{\lambda_j}}\right\}$ tạo thành cơ sở của $H^1_0$ với tích vô hướng $\left<u_x, v_x\right>$.
\end{theorem}

\textit{Chứng minh.}

Phương trình
\begin{align*}
    -\Delta u = \lambda u \Leftrightarrow u_{xx} + \lambda u = 0.
\end{align*}

Phương trình đặc trưng tương ứng
\begin{align*}
    k^2 + \lambda = 0 \Leftrightarrow k^2 = -\lambda.
\end{align*}

Nếu $\lambda \le 0$, ta được $k = \pm\sqrt{-\lambda}$. Khi đó
\begin{align*}
    u(x) = C_1 e^{\sqrt{-\lambda} x} + C_2 e^{-\sqrt{-\lambda} x}.
\end{align*}

Vì $u \in H^1_0$ nên $u(0) = u(1) = 0$, hay thoả hệ phương trình
\begin{align*}
\begin{cases}
    C_1 + C_2 = 0 \\
    C_1 e^{\sqrt{-\lambda}} + C_2 e^{-\sqrt{-\lambda}} = 0
\end{cases}
\end{align*}
suy ra $C_1 = C_2 = 0$, do đó ta loại trường hợp $\lambda \le 0$.

Nếu $\lambda > 0$, khi đó $k = \pm i\sqrt{\lambda}$ và
\begin{align*}
    u(x) = C_1 \cos(\sqrt{\lambda}x) + C_2 \sin(\sqrt{\lambda}x).
\end{align*}

Vì $u \in H^1_0$ nên $u(0) = u(1) = 0$, hay thoả hệ phương trình
\begin{align*}
\begin{cases}
    C_1 = 0, \\
    C_1 \cos(\sqrt{\lambda}) + C_2 \sin(\sqrt{\lambda}) = 0,
\end{cases} \Leftrightarrow \begin{cases}
    C_1 = 0, \\
    C_2 \sin(\sqrt{\lambda}) = 0.
\end{cases}
\end{align*}

Để $u \not\equiv 0$, ta cần $C_2 \ne 0$, khi đó $\sin(\sqrt{\lambda}) = 0$, dẫn đến
\begin{align*}
    \lambda = (j\pi)^2, \: j = 1,2,\dots
\end{align*}

Ta chọn dãy độc lập tuyến tính $\{w_j\}$ có công thức
\begin{align*}
    w_j(x) = C_j \sin(j\pi x), j=1,2,\dots
\end{align*}

Ta thấy
\begin{align*}
    \left<w_i, w_j\right> = C_i C_j \int_0^1 \sin(i\pi x). \sin(j\pi x)\:dx  = 0, \: \forall i \ne j.
\end{align*}
và
\begin{align*}
    \|w_j\|^2 = C_j^2 \int_0^1 \sin^2(j\pi x)\:dx
    = \frac{C_j^2}{2} \int_0^1 \left(1 - \cos(2j \pi x)\right)\:dx \\
    = \frac{C_j^2}{2} \left.\left(x - \frac{\sin(2j \pi x)}{2j \pi}\right)\right|_{x=0}^{x=1}
    = \frac{C_j^2}{2}.
\end{align*}

Để $\|w_j\| = 1$, ta cần $C_j = \sqrt{2}$. Vậy
\begin{align*}
    w_j(x) &= \sqrt{2} \sin(j \pi x), \: j = 1,2,\dots, \\
    \lambda_j &= (j\pi)^2.
\end{align*}

Đạo hàm $w_j$ và chia cho $\sqrt{\lambda_j} = j\pi$, ta được
\begin{align*}
    w_{jx}(x) &= \sqrt{2} j\pi \cos(j\pi x), j=1,2,\dots, \\
    \frac{w_{jx}(x)}{\sqrt{\lambda_j}} &= \sqrt{2} \cos(j\pi x).
\end{align*}

Khi đó
\begin{align*}
    \left<\frac{w_{ix}}{\sqrt{\lambda_i}}, \frac{w_{jx}}{\sqrt{\lambda_j}}\right>
    = 2 \int_0^1 \cos(i\pi x). \cos(j\pi x)\: dx = 0, \forall i \ne j,
\end{align*}
và
\begin{align*}
    \left\|\frac{w_{jx}}{\sqrt{\lambda_j}}\right\|^2
    &= 2 \int_0^1 \cos^2 (j \pi x)\:dx
    = \int_0^1 \left(1 + \cos(2j\pi x)\right)dx \\
    &= \left.\left(x + \frac{\sin(2j\pi x)}{2j\pi}\right)\right|_{x=0}^{x=1}
    = 1.
\end{align*}

Khi đó $\left\{\frac{w_{jx}}{\sqrt{\lambda_j}}\right\}$ là cơ sở Hilbert của $H^1_0$ với tích vô hướng $\left<u_x,v_x\right>$. \qed

\begin{theorem}
    Giải hệ phương trình
    \begin{align*}
    \begin{cases}
        \ddot{c}_{mj}^{(k)}(t)
        + (\sigma \lambda + \lambda_j) \dot{c}_{mj}^{(k)}(t)
        + \lambda_j \mu_m(t) c_{mj}^{(k)}(t)
        = \left<F_m(t), w_j\right>, j = 1,\dots,k, \\
        c_{mj}^{(k)}(0) = \alpha_j^{(k)}, \: \dot{c}_{mj}^{(k)}(0) = \beta_j^{(k)}.
    \end{cases}
    \end{align*}
\end{theorem}

\textit{Chứng minh.}

Từ $u_m^{(k)}(0) = \tilde{u}_{0k}$, ta suy ra
\begin{align*}
    \sum_{j=0}^k c_{mj}^{(k)}(0) w_j = \sum_{j=1}^k \alpha_j^{(k)} w_j,
\end{align*}
dẫn đến $c_{mj}^{(k)}(0) = \alpha_j^{(k)}$. Tương tự, từ $\dot{u}_m^{(k)}(0) = \tilde{u}_{1k}$, ta suy ra
\begin{align*}
    \sum_{j=0}^k \dot{c}_{mj}^{(k)}(0) w_j = \sum_{j=1}^k \beta_j^{(k)} w_j,
\end{align*}
dẫn đến $\dot{c}_{mj}^{(k)}(0) = \beta_j^{(k)}$.

Số hạng thứ nhất, với $\overline{\lambda}_j = \sigma \lambda + \lambda_j$,
\begin{align*}
    \int_0^t \ddot{c}_{mj}^{(k)}(s) e^{\overline{\lambda}_j s}\:ds
    &= \left.\dot{c}_{mj}^{(k)}(s) e^{\overline{\lambda}_j s}\right|_{s=0}^{s=t} - \overline{\lambda}_j \int_0^t \dot{c}_{mj}^{(k)}(s) e^{\overline{\lambda}_j s}\:ds \\
    &= \dot{c}_{mj}^{(k)}(t) e^{\overline{\lambda}_j t} - \dot{c}_{mj}^{(k)}(0) - \overline{\lambda}_j \int_0^t \dot{c}_{mj}^{(k)}(s) e^{\overline{\lambda}_j s}\:ds \\
    &= \dot{c}_{mj}^{(k)}(t) e^{\overline{\lambda}_j t} - \beta_j^{(k)} - \overline{\lambda}_j \int_0^t \dot{c}_{mj}^{(k)}(s) e^{\overline{\lambda}_j s}\:ds \\
    &= \dot{c}_{mj}^{(k)}(t) e^{\overline{\lambda}_j t} - \beta_j^{(k)} - \overline{\lambda}_j \left(\left.c_{mj}^{(k)}(s) e^{\overline{\lambda}_j s}\right|_{s=0}^{s=t} - \overline{\lambda}_j \int_0^t c_{mj}^{(k)}(s) e^{\overline{\lambda}_j s}\:ds\right) \\
    &= \dot{c}_{mj}^{(k)}(t) e^{\overline{\lambda}_j t} - \beta_j^{(k)} - \overline{\lambda}_j \left(c_{mj}^{(k)}(t) e^{\overline{\lambda}_j t} - c_{mj}^{(k)}(0)  - \overline{\lambda}_j \int_0^t c_{mj}^{(k)}(s) e^{\overline{\lambda}_j s}\:ds\right) \\
    &= \dot{c}_{mj}^{(k)}(t) e^{\overline{\lambda}_j t} - \beta_j^{(k)} - \overline{\lambda}_j \left(c_{mj}^{(k)}(t) e^{\overline{\lambda}_j t} - \alpha_j^{(k)}  - \overline{\lambda}_j \int_0^t c_{mj}^{(k)}(s) e^{\overline{\lambda}_j s}\:ds\right) \\
    &= \dot{c}_{mj}^{(k)}(t) e^{\overline{\lambda}_j t} - \beta_j^{(k)} - \overline{\lambda}_j c_{mj}^{(k)}(t) e^{\overline{\lambda}_j t} + \overline{\lambda}_j \alpha_j^{(k)}  + \overline{\lambda}^2_j \int_0^t c_{mj}^{(k)}(s) e^{\overline{\lambda}_j s}\:ds.
\end{align*}

Số hạng thứ hai
\begin{align*}
    \overline{\lambda}_j \int_0^t \dot{c}_{mj}^{(k)}(s) e^{\overline{\lambda}_j s}\:ds
    &= \overline{\lambda}_j \left(\left.c_{mj}^{(k)}(s) e^{\overline{\lambda}_j s}\right|_{s=0}^{s=t} - \overline{\lambda}_j \int_0^t c_{mj}^{(k)}(s) e^{\overline{\lambda}_j s} \:ds\right) \\
    &= \overline{\lambda}_j \left(c_{mj}^{(k)}(t) e^{\overline{\lambda}_j t} - c_{mj}^{(k)}(0) - \overline{\lambda}_j \int_0^t c_{mj}^{(k)}(s) e^{\overline{\lambda}_j s} \:ds\right) \\
    &= \overline{\lambda}_j \left(c_{mj}^{(k)}(t) e^{\overline{\lambda}_j t} - \alpha_j^{(k)} - \overline{\lambda}_j \int_0^t c_{mj}^{(k)}(s) e^{\overline{\lambda}_j s} \:ds\right) \\
    &= \overline{\lambda}_j c_{mj}^{(k)}(t) e^{\overline{\lambda}_j t} - \overline{\lambda}_j \alpha_j^{(k)} - \overline{\lambda}^2_j \int_0^t c_{mj}^{(k)}(s) e^{\overline{\lambda}_j s} \:ds.
\end{align*}

Các số hạng còn lại
\begin{align*}
    \lambda_j \int_0^t \mu_m(s) c_{mj}^{(k)}(s) e^{\overline{\lambda}_j s}\:ds, \\
    \int_0^t \left<F_m(s),w_j\right>e^{\overline{\lambda}_j s}\:ds.
\end{align*}

Phương trình trở thành
\begin{align*}
    \dot{c}_{mj}^{(k)}(t) e^{\overline{\lambda}_j t} - \beta_j^{(k)}
    + \lambda_j \int_0^t \mu_m(s) c_{mj}^{(k)}(s) e^{\overline{\lambda}_j s}\:ds
    = \int_0^t \left<F_m(s),w_j\right>e^{\overline{\lambda}_j s}\:ds,
\end{align*}
tương đương
\begin{align*}
    \dot{c}_{mj}^{(k)}(t) e^{\overline{\lambda}_j t}
    = \beta_j^{(k)}
    - \lambda_j \int_0^t \mu_m(s) c_{mj}^{(k)}(s) e^{\overline{\lambda}_j s}\:ds
    + \int_0^t \left<F_m(s),w_j\right>e^{\overline{\lambda}_j s}\:ds,
\end{align*}
tương đương
\begin{align*}
    \dot{c}_{mj}^{(k)}(t)
    = \beta_j^{(k)} e^{-\overline{\lambda}_j t}
    - \lambda_j e^{-\overline{\lambda}_j t} \int_0^t \mu_m(s) c_{mj}^{(k)}(s) e^{\overline{\lambda}_j s}\:ds
    + e^{-\overline{\lambda}_j t} \int_0^t \left<F_m(s),w_j\right>e^{\overline{\lambda}_j s}\:ds.
\end{align*}

Lấy tích phân theo biến $t$, ta được
\begin{align*}
    c_{mj}^{(k)}(t) - c_{mj}^{(k)}(0)
    = \beta_j^{(k)} \int_0^t e^{-\overline{\lambda}_j r} dr
    - \lambda_j \int_0^t e^{-\overline{\lambda}_j r} dr \int_0^r \mu_m(s) c_{mj}^{(k)}(s) e^{\overline{\lambda}_j s}\:ds  \\
    + \int_0^r e^{-\overline{\lambda}_j r} dr \int_0^r \left<F_m(s),w_j\right>e^{\overline{\lambda}_j s}\:ds,
\end{align*}
tương đương
\begin{align*}
    c_{mj}^{(k)}(t) - \alpha_j^{(k)}
    = - \frac{\beta_j^{(k)}}{\overline{\lambda}_j} \left.e^{-\overline{\lambda}_j r}\right|_{r=0}^{r=t}
    - \lambda_j \int_0^t e^{-\overline{\lambda}_j r} dr \int_0^r \mu_m(s) c_{mj}^{(k)}(s) e^{\overline{\lambda}_j s}\:ds  \\
    + \int_0^r e^{-\overline{\lambda}_j r} dr \int_0^r \left<F_m(s),w_j\right>e^{\overline{\lambda}_j s}\:ds,
\end{align*}
tương đương
\begin{align*}
    c_{mj}^{(k)}(t) - \alpha_j^{(k)}
    = - \frac{\beta_j^{(k)}}{\overline{\lambda}_j} \left(e^{-\overline{\lambda}_j t} - 1\right)
    - \lambda_j \int_0^t e^{-\overline{\lambda}_j r} dr \int_0^r \mu_m(s) c_{mj}^{(k)}(s) e^{\overline{\lambda}_j s}\:ds  \\
    + \int_0^r e^{-\overline{\lambda}_j r} dr \int_0^r \left<F_m(s),w_j\right>e^{\overline{\lambda}_j s}\:ds,
\end{align*}
tương đương
\begin{align*}
    c_{mj}^{(k)}(t) =
    \alpha_j^{(k)} + \frac{\beta_j^{(k)}}{\overline{\lambda}_j} \left(1 - e^{-\overline{\lambda}_j t}\right)
    - \lambda_j \int_0^t e^{-\overline{\lambda}_j r} dr \int_0^r e^{\overline{\lambda}_j s} \mu_m(s) c_{mj}^{(k)}(s)ds  \\
    + \int_0^r e^{-\overline{\lambda}_j r} dr \int_0^r e^{\overline{\lambda}_j s} \left<F_m(s),w_j\right>ds.
\end{align*} \qed

\begin{theorem}
    Chứng tỏ bất đẳng thức sau
    \begin{align*}
        \int_0^t e^{-\overline{\lambda}_j r} dr \int_0^r e^{\overline{\lambda}_j s} f(s)\:ds
        \le \int_0^t dr \int_0^r f(s)\:ds,
    \end{align*}
    với $f(s) \ge 0, \forall s \in (0,t)$.
\end{theorem}

\textit{Chứng minh.} 

Ta có
\begin{align*}
    \int_0^t e^{-\overline{\lambda}_j r} dr \int_0^r e^{\overline{\lambda}_j s} f(s)\:ds
    &= \int_0^t dr \int_0^r e^{-\overline{\lambda}_j (r-s)} f(s)\:ds.
\end{align*}

Vì $s \le r$ nên $e^{-\overline{\lambda}_j (r-s)} \le 1$, kéo theo
\begin{align*}
    \int_0^r e^{-\overline{\lambda}_j (r-s)} f(s)\:ds \le \int_0^r f(s)\:ds
\end{align*}

Khi đó
\begin{align*}
    \int_0^t e^{-\overline{\lambda}_j r} dr \int_0^r e^{\overline{\lambda}_j s} f(s)\:ds
    \le \int_0^t dr \int_0^r f(s)\:ds.
\end{align*} \qed

\begin{theorem}
    Chứng tỏ bất đẳng thức
    \begin{align*}
        \int_0^t dr \int_0^r e^{\gamma s} ds \le \frac{1}{\gamma^2} e^{\gamma t}.
    \end{align*}
\end{theorem}

\textit{Chứng minh.}
\begin{align*}
    \int_0^t dr \int_0^r e^{\gamma s} ds
    &= \frac{1}{\gamma} \int_0^t dr \left.e^{\gamma s}\right|_{s=0}^{s=r}
    = \frac{1}{\gamma} \int_0^r \left(e^{\gamma r} - 1\right)dr \\
    &\le \frac{1}{\gamma} \int_0^r e^{\gamma r} dr
    = \frac{1}{\gamma^2} \left.e^{\gamma r}\right|_{r=0}^{r=t}
    = \frac{1}{\gamma^2} \left(e^{\gamma t} - 1\right) \\
    &\le \frac{1}{\gamma^2} e^{\gamma t}.
\end{align*} \qed

\begin{theorem}
    Hai chuẩn $\|.\|_X$ và $\|.\|_{\gamma,X}$ là hai chuẩn tương đương.
\end{theorem}

\textit{Chứng minh.}
\begin{align*}
    e^{-\gamma T} \|c\|_X
    &= e^{-\gamma T} \sup_{0 \le t \le T} |c(t)|_1
    = \sup_{0 \le t \le T} \left(e^{-\gamma T} |c(t)|_1\right) \\
    &\le \sup_{0 \le t \le T} \left(e^{-\gamma t} |c(t)|_1\right)
    = \|c\|_{\gamma,X},
\end{align*}
ta suy ra $\|c\|_X \le e^{\gamma T} \|c\|_{\gamma,X}$.
\begin{align*}
    \|c\|_{\gamma,X}
    = \sup_{0 \le t \le T} \left(e^{-\gamma t} |c(t)|_1\right)
    \le \sup_{0 \le t \le T} |c(t)|_1
    = \|c\|_X.
\end{align*} \qed

\begin{theorem}
Thay $w_j$ bởi $\dot{u}_m^{(k)}(t) \in H^1_0$ trong
\begin{align*}
    \left<\ddot{u}_m^{(k)}(t), w_j\right>
    + \sigma \left<\dot{u}_{mx}^{(k)}(t), w_{jx}\right>
    + \mu_m(t) \left<u_{mx}^{(k)}(t), w_{jx}\right>
    + \lambda \left<\dot{u}_m^{(k)}(t), w_j\right>
    = \left<F_m(t), w_j\right>,
\end{align*}
ta được
\begin{align*}
    &\frac{1}{2} \frac{d}{dt} \left[ \left\|\dot{u}_m^{(k)}(t)\right\|^2 + \mu_m(t) \left\|u_{mx}^{(k)}(t)\right\|^2 + 2 \lambda \int_0^t \left\|\dot{u}_m^{(k)}(s)\right\|^2  ds + 2 \sigma \int_0^t \left\|\dot{u}_{mx}^{(k)}(s)\right\|^2  ds \right] \\
    &= \frac{1}{2} \mu'_m(t) \left\|u_{mx}^{(k)}(t)\right\|^2 + \left<F_m(t), \dot{u}_m^{(k)}(t)\right>. \notag
\end{align*}
\end{theorem}

\textit{Chứng minh.}

Số hạng thứ nhất
\begin{align*}
    \left<\ddot{u}_m^{(k)}(t), \dot{u}_m^{(k)}(t)\right>
    = \frac{1}{2} \frac{d}{dt} \left\|\dot{u}_m^{(k)}(t)\right\|^2.
\end{align*}

Số hạng thứ hai
\begin{align*}
    \sigma \left<\dot{u}_{mx}^{(k)}(t), \dot{u}_{mx}^{(k)}(t)\right>
    = \sigma \left\|\dot{u}_{mx}^{(k)}(t)\right\|^2
    = \sigma \frac{d}{dt} \int_0^t \left\|\dot{u}_{mx}^{(k)}(s)\right\|^2 ds.
\end{align*}

Số hạng thứ ba
\begin{align*}
    \mu_m(t) \left<u_{mx}^{(k)}(t), \dot{u}_{mx}^{(k}(t)\right>
    &= \frac{1}{2} \mu_m(t) \frac{d}{dt} \left\|u_{mx}^{(k)}(t)\right\|^2 \\
    &= \frac{1}{2} \frac{d}{dt} \left(\mu_m(t) \left\|u_{mx}^{(k)}(t)\right\|^2\right) - \frac{1}{2} \mu'_m(t) \left\|u_{mx}^{(k)}(t)\right\|^2.
\end{align*}

Số hạng thứ tư
\begin{align*}
    \lambda \left<\dot{u}_m^{(k)}(t), \dot{u}_m^{(k)}(t)\right>
    &= \lambda \left\|\dot{u}_m^{(k)}(t)\right\|^2
    = \lambda \frac{d}{dt} \int_0^t \left\|\dot{u}_m^{(k)}(t)\right\|^2 ds.
\end{align*} \qed

\begin{theorem}
Thay $w_j$ bởi $\frac{-1}{\lambda_j} \Delta w_j \in H^1_0$ trong
\begin{align*}
    \left<\ddot{u}_m^{(k)}(t), w_j\right>
    + \sigma \left<\dot{u}_{mx}^{(k)}(t), w_{jx}\right>
    + \mu_m(t) \left<u_{mx}^{(k)}(t), w_{jx}\right>
    + \lambda \left<\dot{u}_m^{(k)}(t), w_j\right>
    = \left<F_m(t), w_j\right>,
\end{align*}
ta được
\begin{align*}
    \left<\ddot{u}_{mx}^{(k)}(t), w_{jx}\right>
    + \sigma \left< \Delta \dot{u}_m^{(k)}(t), \Delta w_j\right>
    + &\mu_m(t) \left<\Delta u_m^{(k)}(t), \Delta w_j\right> \\
    &+ \lambda \left<\dot{u}_{mx}^{(k)}(t), w_{jx}\right>
    = \left<F_{mx}(t), w_{jx}\right>.
\end{align*}
\end{theorem}

\textit{Chứng minh.}

Thay $w_j$ bằng $\frac{-1}{\lambda_j}\Delta w_j$, ta được
\begin{align*}
    \left<\ddot{u}_m^{(k)}(t), \frac{-1}{\lambda_j}\Delta w_j\right>
    &+ \sigma \left<\dot{u}_{mx}^{(k)}(t), \frac{-1}{\lambda_j}\Delta w_{jx}\right>
    + \mu_m(t) \left<u_{mx}^{(k)}(t), \frac{-1}{\lambda_j}\Delta w_{jx}\right> \\
    &+ \lambda \left<\dot{u}_m^{(k)}(t), \frac{-1}{\lambda_j}\Delta w_j\right>
    = \left<F_m(t), \frac{-1}{\lambda_j}\Delta w_j\right>.
\end{align*}

Đơn giản $\lambda_j$, ta được
\begin{align*}
    \left<\ddot{u}_m^{(k)}(t), -\Delta w_j\right>
    + \sigma \left<\dot{u}_{mx}^{(k)}(t), -\Delta w_{jx}\right>
    &+ \mu_m(t) \left<u_{mx}^{(k)}(t), -\Delta w_{jx}\right> \\
    &+ \lambda \left<\dot{u}_m^{(k)}(t), -\Delta w_j\right>
    = \left<F_m(t), -\Delta w_j\right>,
\end{align*}

Số hạng thứ nhất của vế trái
\begin{align*}
    \left<\ddot{u}_m^{(k)}(t), -\Delta w_j\right>
    &= - \left<\ddot{u}_m^{(k)}(t), \Delta w_j\right> \\
    &= - \left(\left.\ddot{u}_m^{(k})(x,t) w_{jx}(x)\right|_{x=0}^{x=1} -\left<\ddot{u}_{mx}^{(k)}(t), w_{jx}\right>\right) \\
    &= \left<\ddot{u}_{mx}^{(k)}(t), w_{jx}\right>,
\end{align*}
do
\begin{align*}
    \ddot{u}_m^{(k)}(0,t) = \sum_{j=0}^k \ddot{c}_{mj}^{(k)}(t) w_j(0) = 0, \\
    \ddot{u}_m^{(k)}(1,t) = \sum_{j=0}^k \ddot{c}_{mj}^{(k)}(t) w_j(1) = 0.
\end{align*}

Số hạng thứ hai của vế trái
\begin{align*}
    \sigma \left<\dot{u}_{mx}^{(k)}(t), -\Delta w_{jx}\right>
    &= -\sigma \left<\dot{u}_{mx}^{(k)}(t), \Delta w_{jx}\right> \\
    &= -\sigma \left(\left.\dot{u}_{mx}^{(k)}(x,t) \Delta w_j(x)\right|_{x=0}^{x=1} - \left<\Delta\dot{u}_m^{(k)}(t), \Delta w_j\right>\right) \\
    &= \sigma \left<\Delta\dot{u}_m^{(k)}(t), \Delta w_j\right>,
\end{align*}
do
\begin{align*}
    \Delta w_j(0) = -\lambda_j w_j(0) = 0, \quad \Delta w_j(1) = -\lambda_j w_j(1) = 0.
\end{align*}

Số hạng thứ ba của vế trái
\begin{align*}
    \mu_m(t) \left<u_{mx}^{(k)}(t), -\Delta w_{jx}\right>
    &= -\mu_m(t) \left<u_{mx}^{(k)}(t), \Delta w_{jx}\right> \\
    &= -\mu_m(t) \left(\left.u_{mx}^{(k)}(x,t) \Delta w_j(x)\right|_{x=0}^{x=1} - \left<\Delta u_m^{(k)}(t), \Delta w_j\right>\right) \\
    &= \mu_m(t) \left<\Delta u_m^{(k)}(t), \Delta w_j\right>.
\end{align*}

Số hạng thứ tư của vế trái
\begin{align*}
    \lambda \left<\dot{u}_m^{(k)}(t), -\Delta w_j\right>
    &= -\lambda \left<\dot{u}_m^{(k)}(t), \Delta w_j\right> \\
    &=  -\lambda \left(\left.\dot{u}_m^{(k)}(x,t) w_{jx}(x)\right|_{x=0}^{x=1} - \left<\dot{u}_{mx}^{(k)}(t), w_{jx}\right>\right) \\
    &= \lambda \left<\dot{u}_{mx}^{(k)}(t), w_{jx}\right>,
\end{align*}
do
\begin{align*}
    \dot{u}_m^{(k)}(0,t) = \sum_{j=0}^k \dot{c}_{mj}^{(k)}(t) w_j(0) = 0, \\
    \dot{u}_m^{(k)}(1,t) = \sum_{j=0}^k \dot{c}_{mj}^{(k)}(t) w_j(1) = 0.
\end{align*}

Số hạng của vế phải
\begin{align*}
    \left<F_m(t), -\Delta w_j\right>
    &= - \left<F_m(t), \Delta w_j\right> \\
    &= -\left(\left.F_m(x,t) w_{jx}(x)\right|_{x=0}^{x=1} - \left<F_{mx}(t), w_{jx}\right>\right) \\
    &= \left<F_{mx}(t), w_{jx}\right>,
\end{align*}
do
\begin{align*}
    F_m(0,t) &= -K u^3_{m-1}(0,t) + f(0,t) = -K.0 + 0 = 0, \\
    F_m(1,t) &= -K u^3_{m-1}(1,t) + f(1,t) = -K.0 + 0 = 0.
\end{align*} \qed

\begin{theorem}
Thay $w_j$ bởi $\dot{u}_m^{(k)}(t) \in H^1_0$ trong
\begin{align*}
    \left<\ddot{u}_{mx}^{(k)}(t), w_{jx}\right>
    + \sigma \left< \Delta \dot{u}_m^{(k)}(t), \Delta w_j\right>
    &+ \mu_m(t) \left<\Delta u_m^{(k)}(t), \Delta w_j\right> \\
    &+ \lambda \left<\dot{u}_{mx}^{(k)}(t), w_{jx}\right>
    = \left<F_{mx}(t), w_{jx}\right>.
\end{align*}
ta được
\begin{align*}
    \frac{1}{2} \frac{d}{dt} \left[ \left\|\dot{u}_{mx}^{(k)}(t)\right\|^2 + \mu_m(t) \left\|\Delta u_{m}^{(k)}(t)\right\|^2 + 2\lambda \int_0^t \left\|\dot{u}_{mx}^{(k)}(s)\right\|^2 ds + 2\sigma \int_0^t \left\|\Delta \dot{u}_{m}^{(k)}(s)\right\|^2 ds \right] \\
    = \frac{1}{2} \mu'_m(t) \left\|\Delta u_{m}^{(k)}(t)\right\|^2 + \left<F_{mx}(t), \dot{u}_{mx}^{(k)}(t)\right>.
\end{align*}
\end{theorem}

\textit{Chứng minh.}

Số hạng thứ nhất của vế trái
\begin{align*}
    \left<\ddot{u}_{mx}^{(k)}(t), \dot{u}_{mx}^{(k)}(t)\right>
    = \frac{1}{2} \frac{d}{dt} \left\|\dot{u}_{mx}^{(k)}(t)\right\|^2.
\end{align*}

Số hạng thứ hai của vế trái
\begin{align*}
    \sigma \left<\Delta \dot{u}_m^{(k)}(t), \Delta \dot{u}_m^{(k)}(t)\right>
    = \sigma \left\|\Delta \dot{u}_m^{(k)}(t)\right\|^2
    = \sigma \frac{d}{dt} \int_0^t \left\|\Delta \dot{u}_m^{(k)}(s)\right\|^2 ds.
\end{align*}

Số hạng thứ ba của vế trái
\begin{align*}
    \mu_m(t) \left<\Delta u_m^{(k)}(t), \Delta \dot{u}_m^{k}(t)\right>
    &= \frac{1}{2} \mu_m(t) \frac{d}{dt} \left\|\Delta u_m^{(k)}(t)\right\|^2 \\
    &= \frac{1}{2} \frac{d}{dt} \left(\mu_m(t) \left\|\Delta u_m^{(k)}(t)\right\|^2\right) - \frac{1}{2} \mu'_m(t) \left\|\Delta u_m^{(k)}(t)\right\|^2.
\end{align*}

Số hạng thứ tư của vế trái
\begin{align*}
    \lambda \left<\dot{u}_{mx}^{(k)}(t), \dot{u}_{mx}^{(k})(t)\right>
    = \lambda \left\|\dot{u}_{mx}^{(k)}(t)\right\|^2
    = \lambda \frac{d}{dt} \int_0^t \left\|\dot{u}_{mx}^{(k)}(s)\right\|^2 ds.
\end{align*}

Số hạng vế phải
\begin{align*}
    \left<F_{mx}(t), \dot{u}_{mx}^{(k)}(t)\right>.
\end{align*} \qed

\begin{theorem}
Thay $w_j$ bởi $\ddot{u}_m^{(k)}(t) \in H^1_0$ trong
\begin{align*}
    \left<\ddot{u}_{mx}^{(k)}(t), w_{jx}\right>
    + \sigma \left< \Delta \dot{u}_m^{(k)}(t), \Delta w_j\right>
    &+ \mu_m(t) \left<\Delta u_m^{(k)}(t), \Delta w_j\right> \\
    &+ \lambda \left<\dot{u}_{mx}^{(k)}(t), w_{jx}\right>
    = \left<F_{mx}(t), w_{jx}\right>.
\end{align*}
ta được
\begin{align*}
    &\frac{d}{dt} \left[ \sigma \left\|\Delta \dot{u}_{m}^{(k)}(t)\right\|^2 + \lambda \left\|\dot{u}_{mx}^{(k)}(t)\right\|^2 + 2 \int_0^t \left\|\ddot{u}_{mx}^{(k)}(t)\right\|^2 \right] \\
    &= -2 \frac{d}{dt} \left[\mu_m(t) \left<\Delta u_m^{(k)}(t), \Delta \dot{u}_m^{(k)}(t)\right>\right] + 2 \mu'_m(t) \left<\Delta u_m^{(k)}(t), \Delta \dot{u}_m^{(k)}(t)\right> \\
    &\quad + 2 \mu_m(t) \left\|\Delta \dot{u}_m^{(k)}(t)\right\|^2 + 2 \left<F_{mx}(t), \ddot{u}_{mx}^{(k)}(t)\right>.
\end{align*}
\end{theorem}

\textit{Chứng minh.}

Số hạng thứ nhất của vế trái
\begin{align*}
    \left<\ddot{u}_{mx}^{(k)}(t), \ddot{u}_{mx}^{(k)}(t)\right>
    = \left\|\ddot{u}_{mx}^{(k)}(t)\right\|^2
    = \frac{d}{dt} \int_0^t \left\|\ddot{u}_{mx}^{(k)}(s)\right\|^2 ds.
\end{align*}

Số hạng thứ hai của vế trái
\begin{align*}
    \sigma \left<\Delta \dot{u}_m^{(k)}(t), \Delta \ddot{u}_m^{(k)}(t)\right>
    = \frac{\sigma}{2} \frac{d}{dt} \left\|\Delta \dot{u}_m^{(k)}(t)\right\|^2.
\end{align*}

Số hạng thứ ba của vế trái
\begin{align*}
    \mu_m(t) \left<\Delta u_m^{(k)}(t), \Delta \ddot{u}_m^{(k)}(t)\right>
    &= \mu_m(t) \frac{d}{dt}\left<\Delta u_m^{(k)}(t), \Delta \dot{u}_m^{(k)}(t)\right> - \mu_m(t) \left\|\Delta \dot{u}_m^{(k)}(t) \right\|^2 \\
    &\begin{aligned}
        =\frac{d}{dt} \left[\mu_m(t) \left<\Delta u_m^{(k)}(t), \Delta \dot{u}_m^{(k)}(t)\right>\right] &- \mu'_m(t) \left<\Delta u_m^{(k)}(t), \Delta \dot{u}_m^{(k)}(t)\right> \\
        &- \mu_m(t) \left\|\Delta \dot{u}_m^{(k)}(t) \right\|^2.
    \end{aligned}
\end{align*}

Số hạng thứ tư của vế trái
\begin{align*}
    \lambda \left< \dot{u}_{mx}^{(k)}(t), \ddot{u}_{mx}^{(k)}(t)\right>
    = \frac{\lambda}{2} \frac{d}{dt} \left\|\dot{u}_{mx}^{(k)}(t)\right\|^2.
\end{align*}

Số hạng vế phải
\begin{align*}
    \left<F_{mx}(t), \ddot{u}_{mx}(t)\right>.
\end{align*} \qed

\begin{theorem}
    Cho $\overline{S}_m^{(k)}(t)$ được cho bởi
    \begin{align*}
        \overline{S}_m^{(k)}(t)= \|\dot{u}_m^{(k)}(t)\|^2_{H^2\cap H^1_0} + \|u_m^{(k)}(t)\|^2_{H^2\cap H^1_0}
        + \int_0^t \left(\|\dot{u}_m^{(k)}(s)\|^2_{H^2\cap H^1_0} + \|\ddot{u}_{mx}^{(k)}(s)\|^2\right)ds,
    \end{align*}
    và $S_m^{(k)}(t)$ được cho bởi
    \begin{align*}
        S_m^{(k)}(t) &= \|\dot{u}_m^{(k)}(t)\|^2
        + (1 + \lambda) \|\dot{u}_{mx}^{(k)}(t)\|^2
        + \sigma \|\Delta \dot{u}_m^{(k)}(t)\|^2
        + \mu_m(t) \left(\|u_{mx}^{(k)}(t)\|^2 + \|\Delta u_m^{(k)}(t)\|^2\right) \\
        &\quad + 2 \lambda \int_0^t \|\dot{u}_m^{(k)}(s)\|^2 ds
        + 2 (\lambda + \sigma) \int_0^t \|\dot{u}_{mx}^{(k)}(s)\|^2 ds \\
        &\quad + 2 \sigma \int_0^t \|\Delta \dot{u}_m^{(k)}(s)\|^2 ds
        + 2 \int_0^t \|\ddot{u}_{mx}^{(k)}(s)\|^2 ds.
    \end{align*}
    Chứng tỏ rằng, với $\sigma_* = \min \{ \lambda, \sigma, 1 \}$, ta có
    \begin{align*}
        S_m^{(k)}(t) \ \ge \ \sigma_* \overline{S}_m^{(k)}(t).
    \end{align*}
\end{theorem}

\textit{Chứng minh.}
\begin{align*}
    S_m^{(k)}(t) &\ge (1 + \lambda) \|\dot{u}_{mx}^{(k)}(t)\|^2
    + \sigma \|\Delta \dot{u}_m^{(k)}(t)\|^2
    + \mu_m(t) \left(\|u_{mx}^{(k)}(t)\|^2 + \|\Delta u_m^{(k)}(t)\|^2\right) \\
    &\quad + 2 (\lambda + \sigma) \int_0^t \|\dot{u}_{mx}^{(k)}(s)\|^2 ds 
    + 2 \sigma \int_0^t \|\Delta \dot{u}_m^{(k)}(s)\|^2 ds
    + 2 \int_0^t \|\ddot{u}_{mx}^{(k)}(s)\|^2 ds \\
    &\ge \sigma_* \Big[\|\dot{u}_{mx}^{(k)}(t)\|^2
    + \|\Delta \dot{u}_m^{(k)}(t)\|^2
    + \|u_{mx}^{(k)}(t)\|^2 + \|\Delta u_m^{(k)}(t)\|^2 \\
    &\qquad\quad + \int_0^t \big(\|\dot{u}_{mx}^{(k)}(s)\|^2 + \|\Delta \dot{u}_m^{(k)}(s)\|^2
    + \|\ddot{u}_{mx}^{(k)}(s)\|^2\big)ds\Big] \\
    &= \sigma_* \Big[\|\dot{u}_m^{(k)}(t)\|^2_{H^2\cap H^1_0} + \|u_m^{(k)}(t)\|^2_{H^2\cap H^1_0}
    + \int_0^t \big(\|\dot{u}_m^{(k)}(s)\|^2_{H^2\cap H^1_0} + \|\ddot{u}_{mx}^{(k)}(s)\|^2\big)ds\Big] \\
    &= \sigma_* \overline{S}_m^{(k)}(t).
\end{align*} \qed

\subsection{Sự hội tụ của dãy lặp}

\begin{theorem}
    Cho $u_m$ thoả bài toán biến phân
    \begin{align*}
    \begin{cases}
        \left<u''_m(t),v\right> + \sigma \left<u'_{mx}(t),v_x\right> + \mu_m(t) \left<u_{mx}(t),v_x\right> + \lambda \left<u'_m(t),v\right> = \left<F_m(t),v\right>, \forall v \in H^1_0, \\
        u_m(0) = u'_m(0) = 0.
    \end{cases}
    \end{align*}
    Đặt $v_m = u_{m+1} - u_m$, thì $v_m$ thoả mãn bài toán biến phân
    \begin{align*}
    \begin{cases}
        \begin{aligned}
            \left<v''_m(t),w\right> &+ \sigma \left<v'_{mx}(t),w_x\right> + \mu_{m+1}(t) \left<v_{mx}(t),w_x\right> \\
            &+ [\mu_{m+1}(t) - \mu_m(t)] \left<u_{mx}(t),w_x\right> + \lambda \left<v'_m(t),w\right> = \left<F_{m+1}(t) - F_m(t),w\right>,
        \end{aligned}\\
        v_m(0) = v'_m(0) = 0.
    \end{cases}
    \end{align*}
\end{theorem}

\textit{Chứng minh.}

Số hạng thứ nhất của vế trái
\begin{align*}
    \left<u''_{m+1}(t),w\right> - \left<u''_m(t),w\right> = \left<u''_{m+1}(t) - u''_m(t),w\right> = \left<w''_m(t),w\right>.
\end{align*}

Số hạng thứ hai của vế trái
\begin{align*}
    \sigma \left<u'_{m+1,x}(t),w_x\right> - \sigma \left<u'_{mx}(t),w_x\right> = \sigma \left<u'_{m+1,x}(t) - u'_{mx}(t),w_x\right> = \sigma \left<v'_{mx}(t),w_x\right>.
\end{align*}

Số hạng thứ ba của vế trái
\begin{align*}
    \mu_{m+1}(t) &\left<u_{m+1,x}(t),w_x\right> - \mu_m(t) \left<u_{mx}(t),w_x\right> \\
    &= \mu_{m+1}(t) \left<v_{mx}(t),w_x\right> + \mu_{m+1}(t)\left<u_{mx},w_x\right> - \mu_m(t) \left<u_{mx}(t),w_x\right> \\
    &= \mu_{m+1}(t) \left<v_{mx}(t),w_x\right> + [ \mu_{m+1}(t) - \mu_m(t)]\left<u_{mx}(t),w_x\right>.
\end{align*}

Số hạng thứ tư của vế trái
\begin{align*}
    \lambda \left<u'_{m+1}(t),w\right> - \lambda \left<u'_m(t),w\right>
    = \lambda \left<u'_{m+1}(t) - u'_m(t),w\right>
    = \lambda \left<v'_m(t),w\right>.
\end{align*}

Số hạng vế phải
\begin{align*}
    \left<F_{m+1}(t),w\right> - \left<F_m(t),w\right> = \left<F_{m+1}(t) - F_m(t),w\right>.
\end{align*}

Điều kiện đầu
\begin{align*}
    v_m(0) = u_{m+1}(0) - u_m(0) = \tilde{u}_0 - \tilde{u}_0 = 0, \\
    v'_m(0) = u'_{m+1}(0) - u'_m(0) = \tilde{u}_1 - \tilde{u}_1 = 0.
\end{align*} \qed

\begin{theorem}
    Lấy $w = v'_m(t) \in H^1_0$ trong bài toán biến phân $v_m$, ta được
    \begin{align*}
        \frac{d}{dt} &\left[ \|v'_m(t)\|^2 + \mu_{m+1}(t)\|v_{mx}(t)\|^2 + 2\sigma\int_0^t \|v'_{mx}(t)\|^2 ds + 2\lambda\int_0^t \|v'_m(s)\|^2 ds \right] \\
        &\begin{aligned}
            = \mu'_{m+1}(t) \|v_{mx}(t)\|^2 &- 2[\mu_{m+1}(t) - \mu_m(t)] \left<u_{mx}(t), v'_{mx}(t)\right> \\ &+ 2\left<F_{m+1}(t) - F_m(t),v'_m(t)\right>.
        \end{aligned}
    \end{align*}
\end{theorem}

\textit{Chứng minh.}

Số hạng thứ nhất của vế trái
\begin{align*}
    \left<v''_m(t), v'_m(t)\right> = \frac{1}{2} \frac{d}{dt} \|v'_m(t)\|^2.
\end{align*}

Số hạng thứ hai của vế trái
\begin{align*}
    \sigma \left<v'_{mx}(t), v'_{mx}(t)\right> = \sigma \|v'_{mx}(t)\|^2 = \sigma \frac{d}{dt} \int_0^t \|v'_{mx}(s)\|^2 ds.
\end{align*}

Số hạng thứ ba của vế trái
\begin{align*}
    \mu_{m+1}(t) \left<v_{mx}(t),v'_{mx}(t)\right>
    &= \frac{1}{2} \mu_{m+1}(t) \frac{d}{dt} \|v_{mx}(t)\|^2 \\
    &= \frac{1}{2} \frac{d}{dt} \left[\mu_{m+1}(t) \|v_{mx}(t)\|^2\right] - \frac{1}{2} \mu'_{m+1}(t) \|v_{mx}(t)\|^2.
\end{align*}

Số hạng thứ tư của vế trái
\begin{align*}
    [\mu_{m+1}(t) - \mu_m(t)] \left<u_{mx}(t), v'_{mx}(t)\right>.
\end{align*}

Số hạng thứ năm của vế trái
\begin{align*}
    \lambda \left<v'_m(t),v'_m(t)\right> = \lambda \|v'_m(t)\|^2 = \lambda \frac{d}{dt}\int_0^t \|v'_m(s)\|^2 ds.
\end{align*}

Số hạng vế phải
\begin{align*}
    \left<F_{m+1}(t) - F_m(t),v'_m(t)\right>.
\end{align*} \qed

\begin{theorem}
    Cho
    \begin{align*}
        \|u_{m+1}(t) - u_m(t)\|_{H_T} \ \le \ k_T \|u_m - u_{m-1}\|_{H_T},
    \end{align*}
    khi đó ta được
    \begin{align*}
        \|u_{m+q}(t) - u_m\|_{H_T} \ \le \ \frac{k_T^m}{1 - k_T} \|u_1 - u_0\|_{H_T},\ \forall q \in \N.
    \end{align*}
\end{theorem}

\textit{Chứng minh.}

Với $q = 1$ ta được
\begin{align*}
    \|u_{m+1} - u_m\|_{H_T}
    \ &\le \ k_T \|u_m - u_{m-1}\|_{H_T} \le k_T^2 \|u_{m-1} - u_{m-2}\|_{H_T}\\
    &\le \ \dots \le k_T^m \|u_1 - u_0\|_{H_T} \le \frac{k_T^m}{1 - k_T} \|u_1 - u_0\|_{H_T}.
\end{align*}

Xét $q \in \N$ bất kỳ, ta có
\begin{align*}
    &\|u_{m+q} - u_m\|_{H_T} \\
    &\le \|u_{m+q} - u_{m+q-1}\|_{H_T} + \|u_{m+q-1} - u_{m+q-2}\|_{H_T} + \dots + \|u_{m+1} - u_m\|_{H_T} \\
    &\le k_T^{m+q-1} \|u_1 - u_0\|_{H_T} + k_T^{m+q-2} \|u_1 - u_0\|_{H_T} + \dots + k_T^m \|u_1 - u_0\|_{H_T} \\
    &= \left( k_T^{q-1} + k_T^{q-2} + \dots + 1\right) k_T^m \|u_1 - u_0\|_{H_T} \\
    &\le \left(\sum_{q=0}^\infty k_T^q\right) k_T^m \|u_1 - u_0\|_{H_T}
    = \frac{k_T^m}{1 - k_T} \|u_1 - u_0\|_{H_T}.
\end{align*} \qed

\begin{theorem}
    Cho dãy $\{u_{m_j}\} \subset W_1(R,T)$ và $u \in H_T$ thoả
    \begin{align*}
    \begin{cases}
        u_{m_j} \to u \text{ yếu* trong } L^\infty(0,T;H^2 \cap H^1_0), \\
        u'_{m_j} \to u' \text{ yếu* trong } L^\infty(0,T;H^2 \cap H^1_0), \\
        u''_{m_j} \to u'' \text{ yếu trong } L^2(0,T;H^1_0), \\
        u \in W(R,T),
    \end{cases}
    \end{align*}
    và $u_{m_j}, u$ thoả phương trình biến phân
    \begin{align*}
        \left<u''(t),w\right> + \sigma \left<u'_x(t),w_x\right> &+ \left(1 + \|u_x(t)\|^2\right) \left<u_x(t),w_x\right> \\
        &+ \lambda \left<u'(t),w\right> = \left<-Ku^3(t) + f(t),w\right>,
    \end{align*}
    với điều kiện đầu cho dãy $\{u_{m_j}\}$
    \begin{align*}
        u_{m_j}(0) = \tilde{u}_0, \ u_{m_j}'(0) = \tilde{u}_1.
    \end{align*}
    Chứng tỏ rằng $u$ thoả điều kiện đầu
    \begin{align*}
        u(0) = \tilde{u}_0, \ u'(0) = \tilde{u}_1.
    \end{align*}
\end{theorem}

\textit{Chứng minh.}

Điều kiện đầu $u(0) = \tilde{u}_0$.

Ta có
\begin{align*}
    \int_0^T \left<u'_{m_j}(t), (T-t)w\right> dt = -T\left<\tilde{u}_0,w\right> + \int_0^T \left<u_{m_j}(t),w\right>dt, \forall j \in \N.
\end{align*}

Cho $j \to \infty$, ta được
\begin{align*}
    \int_0^T \left<u'(t), (T-t)w\right>dt = -T\left<\tilde{u}_0,w\right> + \int_0^T \left<u(t),w\right>dt.
\end{align*}

Mặt khác, ta có
\begin{align*}
    \int_0^T \left<u'(t), (T-t)w\right>dt = -T\left(u(0),w\right> + \int_0^T \left<u(t),w\right>dt.
\end{align*}

Khi đó
\begin{align*}
    \left<u(0),w\right> = \left<\tilde{u}_0,w\right>, \forall w \in H^1_0,
\end{align*}
tương đương $u(0) = \tilde{u}_0$.

Điều kiện đầu $u'(0) = \tilde{u}_1$.

Ta có
\begin{align*}
    \int_0^T \left<u''_{m_j}(t), (T-t)w\right> dt = -T\left<\tilde{u}_1,w\right> + \int_0^T \left<u'_{m_j}(t),w\right>dt, \forall j \in \N.
\end{align*}

Cho $j \to \infty$, ta được
\begin{align*}
    \int_0^T \left<u''(t), (T-t)w\right>dt = -T\left<\tilde{u}_1,w\right> + \int_0^T \left<u('t),w\right>dt.
\end{align*}

Mặt khác, ta có
\begin{align*}
    \int_0^T \left<u''(t), (T-t)w\right>dt = -T\left(u'(0),w\right> + \int_0^T \left<u'(t),w\right>dt.
\end{align*}

Khi đó
\begin{align*}
    \left<u'(0),w\right> = \left<\tilde{u}_1,w\right>, \forall w \in H^1_0,
\end{align*}
tương đương $u'(0) = \tilde{u}_1$. \qed

\begin{theorem}
    Cho $u_1, u_2 \in W_1(R,T)$, đặt $w = u_1 - u_2$. Khi đó $w$ thoả bài toán biến phân
    \begin{align*}
    \begin{cases}
        \begin{aligned}
            \left<w''(t),v\right> &+ \sigma \left<w'_x(t),v_x\right> + \left(1 + \|u_{1x}(t)\|^2\right) \left<w_x(t),v_x\right> + \lambda \left<w'(t),v\right> \\
            &= - \left(\|u_{1x}(t)\|^2 - \|u_{2x}(t)\|^2\right) \left<u_{2x}(t),v_x\right> - K\left<u^3_1(t) - u^3_2(t),v\right>,
        \end{aligned}\\
        w(0) = w'(0) = 0,
    \end{cases}
    \end{align*}
\end{theorem}
với mọi $v \in H^1_0, a.e., t \in (0,T)$.

\textit{Chứng minh.}

Ta có $u_1,u_2$ thoả mãn phương trình biến phân
\begin{align*}
\begin{cases}
    \begin{aligned}
        \left<u''_1(t),v\right> + \sigma \left<u'_{1x}(t),v_x\right> &+ \left(1+\|u_{1x}(t)\|^2\right) \left<u_{1x}(t),v_x\right> \\[0.1cm] &+ \lambda \left<u'_1(t),v\right> = \left<-Ku^3_1(t) + f(t),v\right>,
    \end{aligned} \\[0.4cm]
    \begin{aligned}
         \left<u''_2(t),v\right> + \sigma \left<u'_{2x}(t),v_x\right> &+ \left(1+\|u_{2x}(t)\|^2\right) \left<u_{2x}(t),v_x\right> \\[0.1cm] &+ \lambda \left<u'_2(t),v\right> = \left<-Ku^3_2(t) + f(t),v\right>.
    \end{aligned}
\end{cases}
\end{align*}

Lấy phương trình thứ nhất trừ đi cho phương trình thứ hai, ta được các số hạng như sau.

Số hạng thứ nhất của vế trái
\begin{align*}
    \left<u''_1(t),w\right> - \left<u''_2(t),w\right> = \left<u''_1(t) - u''_2(t),w\right> = \left<w''(t),w\right>.
\end{align*}

Số hạng thứ hai của vế trái
\begin{align*}
    \sigma \left<u'_{1x}(t),v_x\right> - \sigma \left<u'_{2x}(t),v_x\right>
    = \sigma \left<u'_{1x}(t) - u'_{2x}(t),v_x\right> 
    = \sigma \left<w'_x(t),v_x\right>.
\end{align*}

Số hạng thứ ba của vế trái
\begin{align*}
    &\left(1+\|u_{1x}(t)\|^2\right) \left<u_{1x}(t),v_x\right> - \left(1+\|u_{2x}(t)\|^2\right) \left<u_{2x}(t),v_x\right> \\
    &\begin{aligned}
        = \left(1+\|u_{1x}(t)\|^2\right) \left<w_x(t),v_x\right> &+ \left(1+\|u_{1x}(t)\|^2\right) \left<u_{2x}(t),v_x\right> \\ &-\left(1+\|u_{2x}(t)\|^2\right) \left<u_{2x}(t),v_x\right>
    \end{aligned} \\
    &= \left(1+\|u_{1x}(t)\|^2\right) \left<w_x(t),v_x\right> + \left(\|u_{1x}(t)\|^2 - \|u_{2x}(t)\|^2\right) \left<u_{2x}(t),v_x\right>.
\end{align*}

Số hạng thứ tư của vế trái
\begin{align*}
    \lambda \left<u'_1(t),v\right> - \lambda \left<u'_2(t),v\right>
    = \lambda \left<u'_1(t) - u'_2(t),v\right>
    = \lambda \left<w'(t),v\right>.
\end{align*}

Số hạng vế phải
\begin{align*}
     \left<-Ku^3_1(t) + f(t),v\right> - \left<-Ku^3_2(t) + f(t),v\right> = -K \left<u^3_1(t) - u^3_2(t),v\right>.
\end{align*}

Điều kiện đầu
\begin{align*}
    w(0) = u_1(0) - u_2(0) = \tilde{u}_0 - \tilde{u}_0 = 0, \\
    w('0) = u'_1(0) - u'_2(0) = \tilde{u}_1 - \tilde{u}_1 = 0.
\end{align*} \qed

\begin{theorem}
    Lấy $v = w'(t) \in H^1_0$ và lây tích phân theo biến thời gian $t$ cho phương trình biến phân của $w$, ta được
    \begin{align*}
        X(t) &= 2 \int_0^t \left<u_{1x}(s), u'_{1x}(s)\right> \|w_x(s)\|^2 ds \\
        &\quad - 2 \int_0^t \left(\|u_{1x}(s)\|^2 - \|u_{2x}(s)\|^2\right) \left<u_{2x}(s),w'_x(s)\right> ds \\
        &\quad - 2K \int_0^t \left<u^3_1(s) - u^3_2(s), w'(s)\right>ds,
    \end{align*}
    trong đó
    \begin{align*}
        X(t) = \|w'(t)\|^2 + \left(1 + \|u_{1x}(t)\|^2\right) \|w_x(t)\|^2
        + 2\sigma \int_0^t \|w'_x(s)\|^2 ds + 2\lambda \int_0^t \|w'(s)\|^2 ds.
    \end{align*}
\end{theorem}

\textit{Chứng minh.}

Vì $u_1, u_2 \in W_1(R,T)$ nên $u'_1(t), u'_2(t) \in H^1_0$, dẫn đến $w'(t) = u'_1(t) - u'_2(t) \in H^1_0$.

Số hạng thứ nhất của vế trái
\begin{align*}
    \left<w''(t),w'(t)\right> = \frac{1}{2} \frac{d}{dt} \|w'(t)\|^2.
\end{align*}

Số hạng thứ hai của vế trái
\begin{align*}
    \sigma \left<w'_x(t),w'_x(t)\right> = \sigma \|w'_x(t)\|^2 = \sigma \frac{d}{dt} \int_0^t \|w'_x(s)\|^2 ds.
\end{align*}

Số hạng thứ ba của vế trái
\begin{align*}
    \left(1 + \|u_{1x}(t)\|^2\right) &\left<w_x(t),w'_x(t)\right> \\
    &= \frac{1}{2} \left(1 + \|u_{1x}(t)\|^2\right) \frac{d}{dt} \|w_x(t)\|^2 \\
    &= \frac{1}{2} \frac{d}{dt} \left[ \left(1 + \|u_{1x}(t)\|^2 \right) \|w_x(t)\|^2 \right] - \left<u_{1x}(t),u'_{1x}(t)\right> \|w_x(t)\|^2.
\end{align*}

Số hạng thứ tư của vế trái
\begin{align*}
    \lambda \left<w'(t),w'(t)\right> = \lambda \|w'(t)\|^2 = \lambda \frac{d}{dt} \int_0^t \|w'(s)\|^2 ds.
\end{align*}

Số hạng thứ nhất của vế phải
\begin{align*}
    -\left(\|u_{1x}(t)\|^2 - \|u_{2x}(t)\|^2\right) \left<u_{2x}(t),w'_x(t)\right>.
\end{align*}

Số hạng thứ hai của vế phải
\begin{align*}
    -K \left<u^3_1(t) - u^3_2(t),w'(t)\right>.
\end{align*} \qed

\section{Sự tồn tại và duy nhất nghiệm yếu với dữ kiện cho giảm tính trơn}

\begin{theorem}
    Lấy $v = u'_m(t)$ trong
    \begin{align*}
        \left<u''_m(t), v\right>
        + \sigma \left<u'_{mx}(t),v_x\right>
        &+ \left(1 + \|u_{mx}(t)\|^2\right) \left<u_{mx}(t),v_x\right> \\
        &+ \lambda \left<u'_m(t),v\right>
        + K \left<u^3_m(t),v\right>
        = \left<f_m(t),v\right>,
    \end{align*}
    ta được
    \begin{align*}
        S_m(t) = S_m(0) + 2 \int_0^t \left<f_m(s),u'_m(s)\right>ds,
    \end{align*}
    trong đó
    \begin{align*}
        S_m(t) = \|u'_m(t)\|^2
        &+ \int_0^{\|u_{mx}(t)\|^2} (1+z)\:dz
        + \frac{K}{2}\|u(t)\|^4_{L^4} \\
        &+ 2\sigma \int_0^t \|u'_{mx}(s)\|^2 ds
        + 2\lambda \int_0^t \|u'_m(s)\|^2 ds.
    \end{align*}
\end{theorem}

\textit{Chứng minh}.

Số hạng thứ nhất của vế trái
\begin{align*}
    \left<u''_m(t),u'_m(t)\right> = \frac{1}{2} \frac{d}{dt} \|u'_m(t)\|^2.
\end{align*}

Số hạng thứ hai của vế trái
\begin{align*}
    \sigma \left<u'_{mx}(t), u'_{mx}(t)\right> = \sigma \|u'_{mx}(t)\|^2 = \sigma \frac{d}{dt} \int_0^t \|u'_{mx}(s)\|^2 ds.
\end{align*}

Số hạng thứ ba của vế trái
\begin{align*}
    \left(1 + \|u_{mx}(t)\|^2\right) &\left<u_{mx}(t), u'_{mx}(t)\right> \\
    &= \frac{1}{2} \left(1 + \|u_{mx}(t)\|^2\right) \frac{d}{dt} \|u_{mx}(t)\|^2 \\
    &= \frac{1}{2} \frac{d}{dt} \|u_{mx}(t)\|^2 + \frac{1}{2} \|u_{mx}(t)\|^2 \frac{d}{dt} \|u_{mx}(t)\|^2 \\
    &= \frac{1}{2} \frac{d}{dt} \|u_{mx}(t)\|^2 + \frac{1}{4} \frac{d}{dt} \left(\|u_{mx}(t)\|^2\right)^2 \\
    &= \frac{1}{2} \frac{d}{dt} \|u_{mx}(t)\|^2 + \frac{1}{4} \|u_{mx}(t)\|^4 \\
    &= \frac{1}{2} \frac{d}{dt} \left(z + \frac{z^2}{2}\right)_{z=0}^{z=\|u_{mx}(t)\|^2} \\
    &= \frac{1}{2} \frac{d}{dt} \int_0^{\|u_{mx}(t)\|^2} (1+z)\:dz.
\end{align*}

Số hạng thứ tư của vế trái
\begin{align*}
    \lambda \left<u'_m(t),u'_m(t)\right> = \lambda \|u'_m(t)\|^2 = \lambda \frac{d}{dt} \int_0^t \|u'_m(s)\|^2 ds.
\end{align*}

Số hạng thứ năm của vế trái
\begin{align*}
    K \left<u^3_m(t),u'_m(t)\right>
    &= K \int_0^1 u^3_m(x,t) u'_m(x,t)\:dx
    = \frac{K}{4} \int_0^1 \frac{d}{dt}\left(u^4_m(x,t)\right)dx \\
    &= \frac{K}{4} \frac{d}{dt} \int_0^1 u^4_m(x,t)\:dx
    = \frac{K}{4} \frac{d}{dt} \|u_m(t)\|^4_{L^4}.
\end{align*}

Số hạng vế phải
\begin{align*}
    \left<f_m(t), u'_m(t)\right>.
\end{align*} \qed

\begin{theorem}
    Ta đặt
    \begin{align*}
    \begin{cases}
        w_{m,l} = u_m - u_l, \\
        f_{m,l} = f_m - f_l.
    \end{cases}
    \end{align*}
    Với $u_m,u_l$ thoả phương trình biến phân
    \begin{align*}
        \left<u''(t), v\right>
        + \sigma \left<u'_{mx}(t),v_x\right>
        &+ \left(1 + \|u_{x}(t)\|^2\right) \left<u_{x}(t),v_x\right> \\
        &+ \lambda \left<u'(t),v\right>
        + K \left<u^3(t),v\right>
        = \left<f(t),v\right>,
    \end{align*}
    ta suy ra
    \begin{align*}
    \begin{cases}
        \begin{aligned}\left<w''_{m,l}(t),v\right>
        &+ \sigma \left<\nabla w'_{m,l},v_x\right>
        + \left(1 + \|u_{mx}(t)\|^2\right) \left<\nabla w_{m,l}(t),v_x\right> \\[0.1cm]
        &+ \left(\|u_{mx}(t)\|^2 - \|u_{lx}\|^2\right) \left<u_{lx}(t),v_x\right>
        + \lambda \left<w'_{m,l}(t),v\right> \\[0.1cm]
        &+ K \left<u^3_m(t) - u^3_l(t),v\right>
        = \left<f_{m,l},v\right>, \forall v \in H^1_0, \text{a.e.}, t \in (0,T),
        \end{aligned}\\[0.3cm]
        w_{m,l}(0) = \tilde{u}_{0m} - \tilde{u}_{0l}, \  w'_{m,l}(0) = \tilde{u}_{1m} - \tilde{u}_{1l}.
    \end{cases}
    \end{align*}
\end{theorem}

\textit{Chứng minh.}

Ta có $u_m, u_l$ thoả các phương trình biến phân sau
\begin{align*}
\begin{cases}
        \begin{aligned}
            \left<u''_m(t), v\right>
            + \sigma \left<u'_{mx}(t),v_x\right>
            &+ \left(1 + \|u_{mx}(t)\|^2\right) \left<u_{mx} (t),v_x\right> \\[0.1cm]
            &+ \lambda \left<u'_m(t),v\right>
            + K \left<u^3_m(t),v\right>
            = \left<f_m(t),v\right>,
        \end{aligned} \\[0.3cm]
        \begin{aligned}
            \left<u''_l(t), v\right>
            + \sigma \left<u'_{lx}(t),v_x\right>
            &+ \left(1 + \|u_{lx}(t)\|^2\right) \left<u_{lx}(t),v_x\right> \\[0.1cm]
            &+ \lambda \left<u'_l(t),v\right>
            + K \left<u^3_l(t),v\right>
            = \left<f_l(t),v\right>,
        \end{aligned}
\end{cases}
\end{align*}
với điều kiện đầu
\begin{align*}
\begin{cases}
    u_m(0) = \tilde{u}_{0m},\ u'_m(0) = \tilde{u}_{1m}, \\
    u_l(0) = \tilde{u}_{0l},\ u'_l(0) = \tilde{u}_{1l}.
\end{cases}
\end{align*}

Lấy phương trình thứ nhất trừ đi cho phương trình thứ hai, ta được các số hạng như sau.

Số hạng thứ nhất của vế trái
\begin{align*}
    \left<u''_m(t),v\right> - \left<u''_l(t),v\right>
    = \left<u''_m(t) - u''_l(t),v\right>
    = \left<w''_{m,l}(t),v\right>.
\end{align*}

Số hạng thứ hai của vế trái
\begin{align*}
    \sigma \left<u'_{mx}(t),v_x\right> - \sigma \left<u'_{lx}(t),v_x\right>
    = \sigma \left<u'_{mx}(t) - u'_{lx}(t),v_x\right>
    = \sigma \left<\nabla w_{m,l}(t)\right>.
\end{align*}

Số hạng thứ ba của vế trái
\begin{align*}
    &\left(1 + \|u_{mx}(t)\|^2\right) \left<u_{mx}(t),v_x\right> - \left(1 + \|u_{lx}(t)\|^2\right) \left<u_{lx}(t),v_x\right> \\
    &\begin{aligned}= \left(1 + \|u_{mx}(t)\|^2\right) \left<\nabla w_{m,l}(t),v_x\right> &+ \left(1 + \|u_{mx}(t)\|^2\right) \left<u_{lx}(t),v_x\right> \\ &- \left(1 + \|u_{lx}(t)\|^2\right) \left<u_{lx}(t),v_x\right>
    \end{aligned}\\
    &= \left(1 + \|u_{mx}(t)\|^2\right) \left<\nabla w_{m,l}(t),v_x\right> + \left(\|u_{mx}(t)\|^2 - \|u_{lx}(t)\|^2\right) \left<u_{lx}(t),v_x\right>.
\end{align*}

Số hạng thứ tư của vế trái
\begin{align*}
    \lambda \left<u'_m(t),v\right> - \lambda \left<u'_l(t),v\right>
    = \lambda \left<u'_m(t) - u'_l(t),v\right>
    = \lambda \left<w'_{m,l}(t),v\right>.
\end{align*}

Số hạng thứ năm của vế trái
\begin{align*}
    K \left<u^3_m(t),v\right> - K \left<u^3_l(t),v\right>
    = K \left<u^3_m(t) - u^3_l(t),v\right>.
\end{align*}

Số hạng vế phải
\begin{align*}
    \left<f_m(t),v\right> - \left<f_l(t),v\right>
    = \left<f_m(t) - f_l(t),v\right>
    = \left<f_{m,l}(t),v\right>.
\end{align*}

Điều kiện đầu
\begin{align*}
    w_{m,l}(0) = u_m(0) - u_l(0) = \tilde{u}_{0m} - \tilde{u}_{0l}, \\
    w'_{m,l}(0) = u'_m(0) - u'_l(0) = \tilde{u}_{1m} - \tilde{u}_{1l}.
\end{align*} \qed

\begin{theorem}
    Lấy $v = w'_{m,l} = u'_m - u'_l \in H^1_0$ trong biểu thức biến phân của $w_{m,l}$ ta được
    \begin{align*}
        S'_{m,l}(t) &= 2 \left<f_{m,l}(t),w'_{m,l}(t)\right>
        - 2 \|u_{mx}(t)\|^2 \left<\nabla w_{m,l}(t), \nabla w'_{m,l}(t)\right> \\
        &\quad- 2 \left(\|u_{mx}(t)\|^2 - \|u_{lx}(t)\|^2\right) \left<u_{lx}(t), \nabla w'_{m,l}(t)\right>
        - 2 K \left<u^3_m(t) - u^3_l(t), w'_{m,l}(t)\right>,
    \end{align*}
    trong đó
    \begin{align*}
        S_{m,l}(t) = \|w'_{m,l}(t)\|^2 + \|\nabla w_{m,l}(t)\|^2
        + 2 \sigma \int_0^t \|\nabla w'_{m,l}(s)\|^2 ds
        + 2 \lambda \int_0^t \|w'_{m,l}(s)\|^2 ds.
    \end{align*}
\end{theorem}

\textit{Chứng minh.}

Vì $u_m, u_l \in H_T$ nên $u'_m, u'_l \in H^1_0$, dẫn đến $w'_{m,l} \in H^1_0$. 

Số hạng thứ nhất của vế trái
\begin{align*}
    \left<w''_{m,l}(t), w'_{m,l}(t)\right> = \frac{1}{2} \frac{d}{dt} \|w'_{m,l}(t)\|^2.
\end{align*}

Số hạng thứ hai của vế trái
\begin{align*}
    \sigma \left<\nabla w'_{m,l}(t), \nabla w'_{m,l}(t)\right>
    = \sigma \|\nabla w'_{m,l}(t)\|^2
    = \sigma \frac{d}{dt} \int_0^t \|\nabla w'_{m,l}(s)\|^2 ds.
\end{align*}

Số hạng thứ ba của vế trái
\begin{align*}
    \left(1 + \|u_{mx}(t)\|^2\right) &\left<\nabla w_{m,l}(t), \nabla w'_{m,l}(t)\right> \\
    &= \left<\nabla w_{m,l}(t), \nabla w'_{m,l}(t)\right> + \|u_{mx}(t)\|^2 \left<\nabla w_{m,l}(t), \nabla w'_{m,l}(t)\right> \\
    &= \frac{1}{2} \frac{d}{dt} \|\nabla w_{m,l}(t)\|^2 + \|u_{mx}(t)\|^2 \left<\nabla w_{m,l}(t), \nabla w'_{m,l}(t)\right>.
\end{align*}

Số hạng thứ tư của vế trái
\begin{align*}
    \left(\|u_{mx}(t)\|^2 - \|u_{lx}(t)\|^2\right) \left<u_{lx}(t),\nabla w'_{m,l}(t)\right>.
\end{align*}

Số hạng thứ năm của vế trái
\begin{align*}
    \lambda \left<w'_{m,l}(t), w'_{m,l}(t)\right>
    = \lambda \|w'_{m,l}(t)\|^2
    = \lambda \frac{d}{dt} \int_0^t \|w'_{m,l}(s)\|^2 ds.
\end{align*}

Số hạng thứ sáu của vế trái
\begin{align*}
    K \left<u^3_m(t) - u^3_l(t), w'_{m,l}(t)\right>.
\end{align*}

Số hạng vế phải
\begin{align*}
    \left<f_{m,l}(t), w'_{m,l}(t)\right>.
\end{align*}

\section{Tính tắt dần mũ của nghiệm yếu}

\begin{theorem}
Nhân hai vế của phương trình với $u'(x,t)$, với $u$ đủ trơn, sau đó tích phân trên $(0,1)$
\begin{align*}
    u_{tt} - \sigma u_{txx} - \left(1 + \|u_x(t)\|^2\right) u_{xx} + Ku^3 + \lambda u_t = f(x,t), 0 < x < 1, 0 < t < T.
\end{align*}
ta được
\begin{align*}
    \frac{d}{dt}\left[\frac{1}{2} \|u'(t)\|^2 + \frac{1}{2} \|u_x(t)\|^2 + \frac{1}{4} \|u_x(t)\|^4 + \frac{K}{4} \|u(t)\|^4_{L^4}\right] \\
    = -\lambda \|u'(t)\|^2 - \sigma \|u'_x(t)\|^2 + \left<f(t), u'(t)\right>.
\end{align*}
\end{theorem}

\textit{Chứng minh.}

Số hạng thứ nhất của vế trái
\begin{align*}
    \int_0^1  u''(x,t) u'(x,t)\:dx = \left<u''(t), u'(t)\right> = \frac{1}{2} \frac{d}{dt} \|u'(t)\|^2.
\end{align*}

Số hạng thứ hai của vế trái
\begin{align*}
    -\sigma \int_0^1 u_{xx}(x,t) u'(x,t)\:dx &= \sigma \left<u_{xx}(t), u'(t)\right> \\
    &= -\sigma \left(\left.u'_x(x,t) u'(x,t)\right|_{x=0}^{x=1} - \left<u'_x(t), u'_x(t)\right>\right) \\
    &= \sigma \left<u'_x(t), u'_x(t)\right> \\
    &= \sigma \|u'_x(t)\|^2.
\end{align*}

Số hạng thứ tư của vế trái
\begin{align*}
    K \int_0^1 u^3(x,t) u'(x,t)\:dx &= K \int_0^t \frac{1}{4} \frac{d}{dt} \left(u^4(x,t)\right)dx \\
    &= \frac{K}{4} \frac{d}{dt} \int_0^1 u^4(x,t)\:dx = \frac{K}{4} \frac{d}{dt} \|u(t)\|^4_{L^4}.
\end{align*}

Số hạng thứ ba của vế trái
\begin{align*}
    -\left(1 + \|u_x(t)\|^2\right) &\int_0^1 u_{xx}(x,t) u'(x,t)\:dx \\
    &= -\left(1 + \|u_x(t)\|^2\right) \left<u_{xx}(t), u'(t)\right> \\
    &= -\left(1 + \|u_x(t)\|^2\right) \left(\left.u_x(x,t) u'(x,t)\right|_{x=0}^{x=1} - \left<u_x(t), u'_x(t)\right>\right) \\
    &= \left(1 + \|u_x(t)\|^2\right) \left<u_x(t), u'_x(t)\right> \\
    &= \frac{1}{2} \left(1 + \|u_x(t)\|^2\right) \frac{d}{dt} \|u_x(t)\|^2 \\
    &= \frac{1}{2} \frac{d}{dt} \|u_x(t)\|^2 + \frac{1}{2} \|u_x(t)\|^2 \frac{d}{dt} \|u_x(t)\|^2 \\
    &= \frac{1}{2} \frac{d}{dt} \|u_x(t)\|^2 + \frac{1}{4} \frac{d}{dt} \left( \|u_x(t)\|^2 \right)^2 \\
    &= \frac{1}{2} \frac{d}{dt} \|u_x(t)\|^2 + \frac{1}{4} \frac{d}{dt} \|u_x(t)\|^4 \\
    &= \frac{1}{2} \frac{d}{dt} \int_0^{\|u_x(t)\|^2} (1+z)\:dz.
\end{align*}

Số hạng thứ năm của vế trái
\begin{align*}
    \lambda \int_0^1 u'(x,t) u'(x,t)\:dx = \lambda \left<u'(t),u'(t)\right> = \lambda \|u'(t)\|^2.
\end{align*}

Số hạng của vế phải
\begin{align*}
    \int_0^1 f(x,t) u'(x,t)\:dx = \left<f(t),u'(t)\right>.
\end{align*} \qed

\begin{theorem}
Nhân hai vế của phương trình với $u(x,t)$, với $u$ đủ trơn, sau đó tích phân trên $(0,1)$
\begin{align*}
    u_{tt} - \sigma u_{txx} - \left(1 + \|u_x(t)\|^2\right) u_{xx} + Ku^3 + \lambda u_t = f(x,t), 0 < x < 1, 0 < t < T.
\end{align*}
ta được
\begin{align*}
    &\frac{d}{dt}\left[\left<u(t),u'(t)\right> + \frac{\lambda}{2}\|u(t)\|^2 + \frac{\sigma}{2}\|u_x(t)\|^2\right] \\
    &\quad = \|u'(t)\|^2 - \|u_x(t)\|^2 - \|u_x(t)\|^4 - K\|u(t)\|^4_{L^4} + \left<f(t),u(t)\right>.
\end{align*}
\end{theorem}

\textit{Chứng minh.}

Số hạng thứ nhất của vế trái
\begin{align*}
    \int_0^1 u''(x,t) u(x,t)\:dx &= \left<u''(t),u(t)\right> \\
    &= \frac{d}{dt} \left<u(t), u'(t)\right> - \left<u'(t),u'(t)\right>
    = \frac{d}{dt} \left<u(t), u'(t)\right> - \|u'(t)\|^2.
\end{align*}

Số hạng thứ hai của vế trái
\begin{align*}
    -\sigma \int_0^1 u'_{xx}(x,t) u(x,t)\:dx
    &= -\sigma \left<u'_{xx}(t), u(t)\right> \\
    &= -\sigma \left(\left.u'_x(x,t)u(x,t)\right|_{x=0}^{x=1} - \left<u'_x(t), u_x(t)\right>\right) \\
    &= \sigma \left<u'_x(t), u_x(t)\right>
    = \frac{\sigma}{2} \frac{d}{dt} \|u_x(t)\|^2.
\end{align*}

Số hạng thứ ba của vế trái
\begin{align*}
    -\left(1 + \|u_x(t)\|^2\right) &\int_0^1 u_{xx}(x,t)u(x,t)\:dx \\
    &= - \left(1 + \|u_x(t)\|^2\right) \left<u_{xx}(t),u(t)\right> \\
    &= - \left(1 + \|u_x(t)\|^2\right) \left(\left.u_x(x,t)u(x,t)\right|_{x=0}^{x=1} - \left<u_x(t), u_x(t)\right>\right) \\
    &= \left(1 + \|u_x(t)\|^2\right) \left<u_x(t), u_x(t)\right>
    = \left(1 + \|u_x(t)\|^2\right) \|u_x(t)\|^2 \\
    &= \|u_x(t)\|^2 + \|u_x(t)\|^4.
\end{align*}

Số hạng thứ tư của vế trái
\begin{align*}
    K \int_0^1 u^3(x,t) u(x,t)\:dx = K \int_0^1 u^4(x,t)\:dx = K \|u(t)\|^4_{L^4}.
\end{align*}

Số hạng thứ năm của vế trái
\begin{align*}
    \lambda \int_0^1 u'(x,t) u(x,t)\:dx = \lambda \left<u'(t), u(t)\right> = \frac{\lambda}{2} \frac{d}{dt} \|u(t)\|^2.
\end{align*}

Số hạng của vế phải
\begin{align*}
    \int_0^1 f(x,t)u(x,t)\:dx = \left<f(t), u(t)\right>.
\end{align*} \qed

\begin{theorem}
    Tích phân hai vế của
    \begin{align*}
        \mathcal{L}'(t) \le -\overline{\gamma} \mathcal{L}(t) + \tilde{C}_1 e^{-2 \eta_* t}, \forall t \ge 0,
    \end{align*}
    ta được
    \begin{align*}
        \mathcal{L}(t) \le \left(\mathcal{L}(0) + \frac{\tilde{C}_1}{2\eta_* - \overline{\gamma}}\right) e^{-\gamma t}, \forall t \ge 0.
    \end{align*}
\end{theorem}

\textit{Chứng minh.}

Nhân hai vế bất phương trình với $e^{\overline{\gamma}t}$, ta được
\begin{align*}
    e^{\overline{\gamma}t} \mathcal{L}'(t) + \overline{\gamma} e^{\overline{\gamma}t} \mathcal{L}(t) \ \le \ \tilde{C}_1 e^{-(2\eta_* - \overline{\gamma})t},
\end{align*}
tương đương
\begin{align*}
    \frac{d}{dt}\left(e^{\overline{\gamma}t} \mathcal{L}(t)\right) \ \le \ \tilde{C}_1 e^{-(2\eta_* - \overline{\gamma})t}.
\end{align*}

Tích phân hai vế ta được
\begin{align*}
    e^{\overline{\gamma}t} \mathcal{L}(t) - \mathcal{L}(0) \ \le \ \tilde{C}_1 \int_0^t e^{-(2\eta_* - \overline{\gamma})s}ds,
\end{align*}
tương đương
\begin{align*}
    e^{\overline{\gamma}t} \mathcal{L}(t) \ \le \ \mathcal{L}(0) + \frac{\tilde{C}_1}{2\eta_* - \overline{\gamma}} \left(1 - e^{-(2\eta_* - \overline{\gamma})t}\right) \ \le \ \mathcal{L}(0) + \frac{\tilde{C}_1}{2\eta_* - \overline{\gamma}}.
\end{align*}
tương đương
\begin{align*}
    \mathcal{L}(t) \ \le \ \left(\mathcal{L}(0) + \frac{\tilde{C}_1}{2\eta_* - \overline{\gamma}}\right) e^{-\overline{\gamma}t}.
\end{align*} \qed

\end{document}
