\documentclass[9pt]{beamer}
\usepackage[utf8]{vietnam}
%----------Packages----------
\usepackage{enumitem}
\usepackage{amsmath}
\usepackage{amssymb}
\usepackage{framed,color}
\usepackage{esint}
\usepackage{amsthm}
%\usepackage{amsrefs}
\usepackage{dsfont}
\usepackage{color}
\usepackage{caption}
\usepackage{subcaption}
\usepackage[all]{xy}
\usepackage[mathscr]{eucal}
\usepackage{verbatim}  %%includes comment environmenthttps://www.overleaf.com/7333733538cgxrxfpzqxhk
\usepackage{hyperref}
\usepackage[scr]{rsfso}
\usepackage{tikz}
\usepackage{tikz-cd}
\usepackage{graphicx}

\usepackage[
backend=biber,
style=alphabetic,
sorting=ynt
]{biblatex}
\addbibresource{references.bib}


%---------math operators------
\newcommand{\norm}[1]{\left\lVert#1\right\rVert}
\DeclareMathOperator\supp{supp}
\DeclareMathOperator{\Hom}{Hom}
\DeclareMathOperator{\sets}{\mathbf{Sets}}
\DeclareMathOperator{\topbf}{\mathbf{Top}}
\DeclareMathOperator{\topbfast}{\mathbf{Top_\ast}}
\DeclareMathOperator{\grp}{\mathbf{Groups}}
\DeclareMathOperator{\obj}{obj}
\DeclareMathOperator{\id}{Id}
\DeclareMathOperator{\esssup}{esssup}
\DeclareMathOperator{\sgn}{sgn}

\mode<presentation>
{
  \usetheme{Warsaw}      % or try Darmstadt, Madrid, Warsaw, ...
  \usecolortheme{whale} % or try albatross, beaver, crane, ...
  \usefonttheme{default}  % or try serif, structurebold, ...
  \setbeamertemplate{navigation symbols}{}
  \setbeamertemplate{caption}[numbered]
}

\usepackage[english]{babel}
\usepackage[utf8x]{inputenc}
\AtBeginSection[]
{
  \begin{frame}
    \frametitle{Mục lục}
    \tableofcontents[currentsection]
  \end{frame}
}

\title[Tính chất nghiệm của bài toán Dirichet cho phương trình Kirchhoff phi tuyến]{Tính chất nghiệm của bài toán Dirichet cho phương trình Kirchhoff phi tuyến chứa số hạng tắt dần mạnh}
\author[Đỗ Sỹ Hưng]{\large Đỗ Sỹ Hưng \\[0.2cm] \normalsize \textbf{GVHD:} TS. Nguyễn Thành Long}
\institute{Trường Đại học Khoa học Tự nhiên, ĐHQG TP.HCM}
\date{Ngày 10 tháng 07 năm 2023}


\begin{document}

\begin{frame}
  \titlepage
\end{frame}

%---------------------------------------------------
%---------------------------------------------------

% \begin{frame}[plain]
% 	\tableofcontents
% \end{frame}
% Uncomment these lines for an automatically generated outline.
\begin{frame}{Mục lục}
 \tableofcontents
\end{frame}


%---------------------------------------------------
%---------------------------------------------------
\section{Phần tổng quan}

\subsection{Phương trình Kirchoff phi tuyến}

\begin{frame}
    \begin{block}{Phương trình Kirchoff phi tuyến}
    Ta xét bài toán sau
    \begin{align}
        \begin{aligned}
        u_{tt} - \sigma u_{txx} - \left(1 + \textcolor{blue}{\|u_x(t)\|^2}\right) u_{xx} &+ Ku^3 + \lambda u_t  \\
        &= f(x,t),\: 0 < x < 1,\: 0 < t < T,
        \end{aligned}\label{maineqn}
    \end{align}
    với điều kiện Dirichet thuần nhất
    \begin{align}
        u(0,t) = u(1,t) = 0,
    \end{align}
    và điều kiện đầu
    \begin{align}
        u(x,0) = \tilde{u}_0(x);\ u_t(x,0) = \tilde{u}_1(x),
    \end{align}
    trong đó $K, \sigma, \lambda > 0$ là các hằng số thực cho trước, hàm $f, \tilde{u}_0, \tilde{u}_1$ là các hàm cho trước.
    \end{block}
\end{frame}

\subsection{Phương trình biến phân}

\begin{frame}
    \begin{block}{Phương trình biến phân}
    Nhân hai vế của phương trình $\eqref{maineqn}_1$ với $v \in H^1_0$, ta được phương trình biến phân sau
    \begin{align} \label{vareqn}
        \left< u''(t), v \right>
        + \sigma \left< u'_x(t), v_x \right>
        &+ \left(1 + \|u_x(t)\|^2\right) \left< u_x(t), v_x \right> \\
        &+ K \left< u^3(t), v \right> + \lambda \left< u'(t), v \right>
        = \left< f(t), v \right>, \notag
    \end{align}
    hầu hết $t \in (0,T)$, với điều kiện đầu
    \begin{align}
        u(0) = \tilde{u}_0;\ u'(0) = \tilde{u}_1.
    \end{align}
    \end{block}
\end{frame}

\subsection{Các bài toán liên quan đến khoá luận}

\begin{frame}
    \begin{block}{Bài toán 1}
    Trong \cite{4}, các tác giả đã chứng minh được sự tồn tại duy nhất và khai triển tiệm cận của nghiệm của phương trình
    \begin{align*}
    \begin{cases}
        u_{tt} - \left(b_0 + B\left(\textcolor{blue}{\|u_x\|^2}\right)\right) u_{xx} = f(x,t,u,u_x,u_t),\ 0 < x < 1,\ 0 < t < T, \\
        u(0,t) = u(1,t) = 0, \\
        u(x,0) = \tilde{u}_0(x),\ u_t(x,0) = \tilde{u}_1(x).
    \end{cases}
    \end{align*}
    trong đó $b_0 > 0$ là hằng số thực và $B, f, \tilde{u}_0, \tilde{u}_1$ là các hàm cho trước.
    \end{block}

    \begin{block}{Bài toán 2}
    Trong \cite{6}, Triết đã thu được các kết quả tương tự cho bài toán giá trị biên và ban đầu cho phương trình phi tuyến
    \begin{align*}
    \begin{cases}
        u_{tt} - \frac{\partial}{\partial x} \left(B\left(x,t,u, \|u\|^2, \textcolor{blue}{\|u_x\|^2}\right) u_x\right) = f(x,t,u,u_x,u_t), \ 0 < x < 1,\ 0 < t < T, \\
        \textcolor{teal}{u_x(0,t) - h_0 u(0,t) = g_0(t),} \\
        \textcolor{teal}{u(1,t) = g_1(t),} \\
        u(x,0) = \tilde{u}_0(x),\ u_t(x,0) = \tilde{u}_1(x).
    \end{cases}
    \end{align*}
    trong đó $h_0 \ge 0$ là hằng số thực và $B, f, g_0, g_1, \tilde{u}_0, \tilde{u}_1$ là các hàm cho trước.
    \end{block}
\end{frame}

\section{Sự tồn tại và duy nhất nghiệm yếu}

\subsection{Giả thiết cho các hàm $\tilde{u}_0, \tilde{u}_1$ và $f$}

\begin{frame}
    \begin{block}{Giả thiết cho các hàm $\tilde{u}_0, \tilde{u}_1$ và $f$}
    $(\text{H}_1)$ $\tilde{u}_0, \tilde{u}_1 \in H^2 \cap H^1_0$,

    $(\text{H}_2)$ $f \in C^1([0,1]\times[0,T^*])$ thoả mãn
    \begin{align*}
        f(0,t) = f(1,t) = 0, \ \forall t \in [0,T^*].
    \end{align*}
    \end{block}

    \begin{exampleblock}{Tập hợp và không gian hàm $W_T, W(R,T)$ và $W_1(R,T)$}
    Với $0 < T < T^*$ và $R > 0$, ta có không gian hàm $W_T$
    \begin{align}
        W_T = \{ v \in L^\infty (0,T;H^2 \cap H^1_0) \colon v' \in L^\infty (0,T; H^2 \cap H^1_0), v'' \in L^2(0,T; H^1_0) \},
    \end{align}
    với chuẩn
    \begin{align}
        \|v\|_{W_T} = \max \{ \|v\|_{L^\infty(0,T;H^2 \cap H^1_0)} ; \|v'\|_{L^\infty(0,T;H^2 \cap H^1_0)}; \|v''\|_{ L^2(0,T;H^1_0)} \}.
    \end{align}
    Ngoài ra, ta định nghĩa hai tập hợp của $W_T$ như sau
    \begin{align}
        W(R,T) &= \{ v \in W_T \colon \|v\|_{W_T} \le R \}, \\
        W_1(R,T) &= \{ v \in W(R,T) \colon v'' \in L^\infty(0,T;L^2) \}. \notag
    \end{align}
    \end{exampleblock}
\end{frame}

\subsection{Sự tồn tại dãy trong $W_1(R,T)$}

\begin{frame}
    \begin{block}{Định lý 3.1. Sự tồn tại dãy trong $W_1(R,T)$}
    Giả sử $(\text{H}_1)$-$(\text{H}_2)$ đúng. Khi đó, tồn tại các hằng số $R > 0, T > 0$ sao cho tồn tại dãy $\{u_m\} \subset W_1(R,T)$ xác định bởi $u_0 \equiv 0$ và nếu $u_{m-1} \in W_1(R,T)$, thì $u_m \in W_1(R,T)$ là nghiệm của Bài toán \eqref{vareqn}, cụ thể $u_m$ là nghiệm của bài toán
    \begin{align} \label{eqn31}
    \begin{cases}
        \begin{aligned}
            \left< u''_m(t), v \right>
        &+ \sigma \left< u'_{mx}(t), v_x \right>
        + \left(1 + \textcolor{blue}{\|\nabla u_{m-1}(t)\|^2}\right) \left< u_{mx}(t), v_x \right> \\
        &+ K \left< \textcolor{blue}{u^3_{m-1}(t)}, v \right> + \lambda \left< u'_m(t), v \right>
        = \left< f(t), v \right>,\ \forall v \in H^1_0.
        \end{aligned} \\
        u_m(0) = \tilde{u}_0; \  u'_m(0) = \tilde{u}_1.
    \end{cases}
    \end{align}
    \end{block}

    \begin{exampleblock}{Sơ lược chứng minh}
        Cho $u_{m-1} \in W_1(R,T)$
        \begin{itemize}
            \item[(i)] Chứng minh $u_m \in W(R,T)$ bằng phương pháp xấp xỉ Faedo-Galerkin liên kết với thuật giải xấp xỉ tuyến tính cùng với các kỹ thuật đánh giá tiên nghiệm.
            \item[(ii)] Chứng minh $u_m \in W_1(R,T)$ và thoả Bài toán \eqref{eqn31} bằng phương pháp hội tụ yếu dựa vào tính compact.
        \end{itemize}
    \end{exampleblock}
\end{frame}

\subsection{Không gian hàm $H_T$}

\begin{frame}
    \begin{exampleblock}{Không gian hàm $H_T$}
    Không gian hàm $H_T$
    \begin{align}
        H_T = \left\{ v \in C^0([0,T]; H^1_0) \cap C^1([0,T]; L^2) \colon v' \in L^2(0,T;H^1_0) \right\},
    \end{align}
    với chuẩn
    \begin{align}
        \|v\|_{H_T} = \|v\|_{C^0([0,T]; H^1_0)} + \|v\|_{C^1([0,T]; L^2)} + \|v'\|_{L^2(0,T;H^1_0)},
    \end{align}
    hoặc chuẩn tương đương
    \begin{align}
        \|v\|_{H_T} = \|v\|_{C^0([0,T]; H^1_0)} + \|v'\|_{C^0([0,T]; L^2)} + \|v'\|_{L^2(0,T;H^1_0)}.
    \end{align}
    \end{exampleblock}
\end{frame}

\subsection{Sự tồn tại và duy nhất nghiệm trong $H_T$}

\begin{frame}
     \begin{block}{Định lý 3.3. Sự tồn tại và duy nhất nghiệm trong $H_T$}
        Giả sử $(\text{H}_1)$-$(\text{H}_2)$ đúng. Khi đó, hai hằng số $T > 0, R > 0$ được xác định ở Định lý 3.1 sao cho
        \begin{itemize}
            \item[(1)] Bài toán \eqref{vareqn} có duy nhất nghiệm $u \in W_1(R,T)$,
            \item[(2)] Dãy quy nạp tuyến tính $\{u_m\}$ hội tụ về $u \in H_T$. Hơn nữa, ta có đánh giá sai số
        \end{itemize}
        \begin{align}
            \|u_m - u\|_{H_T} \le \frac{R}{1 - k_T} k_T^m, \ \forall m \in \mathbb{N},
        \end{align}
        với $\color{blue} 0 < k_T < 1$ phụ thuộc $T, \lambda, \sigma, f, \tilde{u}_0, \tilde{u}_1$.
    \end{block}    
\end{frame}

\section{Sự tồn tại và duy nhất nghiệm yếu với dữ kiện cho giảm tính trơn}

\subsection{Sự tồn tại và duy nhất nghiệm trong $H_T$}

\begin{frame}
    \begin{block}{Giả thiết chương 3}
    $(\text{H}_1)$ $\tilde{u}_0, \tilde{u}_1 \in H^2 \cap H^1_0$,

    $(\text{H}_2)$ $f \in C^1([0,1]\times[0,T^*])$ thoả mãn
    \begin{align*}
        f(0,t) = f(1,t) = 0, \ \forall t \in [0,T^*].
    \end{align*}
    \end{block}

    \begin{block}{Giả thiết với dữ kiện giảm tính trơn}
    $(\tilde{\text{H}}_1)$ $(\tilde{u}_0, \tilde{u}_1) \in H^1_0 \times L^2$,

    $(\tilde{\text{H}}_2)$ $f \in L^2(Q_T); Q_T = (0,1) \times (0, T)$.
    \end{block}

    \begin{block}{Định lý 4.1. Sự tồn tại và duy nhất nghiệm trong $H_T$}
    Cho $T > 0, K > 0, \lambda > 0$ và các giả thiết $(\tilde{\text{H}}_1)$-$(\tilde{\text{H}}_2)$ được thoả. Khi đó, bài toán \eqref{vareqn} có duy nhất một nghiệm $u \in H_T$.
    \end{block}
\end{frame}

\section{Tính tắt dần mũ của nghiệm}

\begin{frame}
    \begin{block}{Giả thiết về tính tắt dần mũ của hàm $f$}
    $(\tilde{\text{H}}_1)$ $(\tilde{u}_0, \tilde{u}_1) \in H^1_0 \times L^2$,

    $(\tilde{\text{H}}_3)$ $f \in L^2(\mathbb{R}_{+}; L^2)$ sao cho tồn tại $C_* > 0, \gamma_* > 0$ thoả mãn
    \begin{align}
        \|f(t)\| \le C_* e^{-\gamma_* t}, \  \forall t \ge 0.
    \end{align}
    \end{block}

    \begin{block}{Tính tắt dần mũ của nghiệm}
    Hàm $u$ là nghiệm của Bài toán \eqref{vareqn} và tồn tại các số dương $C, \gamma$ sao cho
    \begin{align}
        \|u'(t)\|^2 + \|u_x(t)\|^2 \le C e^{-\gamma t}, \ \forall t \ge 0.
    \end{align}
    Khi đó hàm, $u$ có tính chất tắt dần mũ.
    \end{block}

    \begin{block}{Định lý 5.4. Tính tắt dần mũ của nghiệm}
        Cho trước $K > 0, \lambda > 0$. Với các giả thiết $(\tilde{\text{H}}_1)$-$(\tilde{\text{H}}_3)$, bài toán \eqref{vareqn} có duy nhất một nghiệm $\color{blue} u \in C^0(\mathbb{R}_{+}; L^2) \cap C^1(\mathbb{R}_{+}; L^2)$ sao cho $\color{teal} u' \in L^2_{\text{loc}}(0,\infty; H^1_0)$ thoả tính tắt dần mũ, nghĩa là, tồn tại hai hằng số dương $\overline{C}, \overline{\gamma}$ sao cho
        \begin{align}
            \|u'(t)\|^2 + \|u_x(t)\|^2 + \|u_x(t)\|^4 + \|u(t)\|^4_{L^4} \le \overline{C} e^{-\overline{\gamma} t}, \  \forall t \ge 0.
        \end{align}
    \end{block}
\end{frame}

\begin{frame}{Kết luận}
    \textbf{Chương 3:} \\
    Với các điều kiện $f, \tilde{u}_0, \tilde{u}_1$, ta chứng minh được sự tồn tại và duy nhất nghiệm \textbf{địa phương} của Bài toán biến phân \eqref{vareqn}.
    \vspace{0.3cm}
    
    \textbf{Chương 4:} \\
    Khi các dữ kiện được cho giảm tính trơn, Sự tồn tại và duy nhất nghiệm \textbf{toàn~cục} của Bài toán biến phân \eqref{vareqn}.
     \vspace{0.3cm}
     
    \textbf{Chương 5:} \\
    Cho giả thiết tắt dần mũ của hàm $f$, ta chứng minh được tính tắt dần của nghiệm.
\end{frame}

\begin{frame}
    \frametitle{Tài liệu tham khảo}
    \begin{thebibliography}{1}
\bibitem[1]{Bre} H. Brézis, \textit{Functional Analysis, Sobolev Spaces
and Partial Differential Equations}, Springer New York Dordrecht Heidelberg
London, 2010.

\bibitem[2]{Lions} J.L. Lions, \textit{Quelques m\'{e}thodes de r\'{e}%
solution des probl\`{e}mes aux limites nonlin\'{e}aires}, Dunod; Gauthier-
Villars, Paris. 1969.

\bibitem[3]{3} Nguyen Thanh Long, Le Thi Phuong Ngoc, \textit{On nonlinear
boundary value problems for nonlinear wave equations}, Vietnam J. Math. 
\textbf{37 }(2-3) (2009) 141-178\textbf{.}

\bibitem[4]{4} Nguyen Thanh Long, Alain Pham Ngoc Dinh, Tran Ngoc Diem, 
\textit{Linear recursive schemes and asymptotic expansion associated with
the Kirchhoff-Carrier operator}, J. Math. Anal. Appl. \textbf{267} (1)
(2002) 116-134.

\bibitem[5]{5} Le Thi Phuong Ngoc, Nguyen Thanh Long,
\textit{Existence and exponential decay for a nonlinear wave equation with 
a nonlocal boundary condition}, Communications on Pure and Applied Analysis,
\textbf{12} (5) (2013) 2001-2029.

\bibitem[6]{6} Nguyen Anh Triet, Le Thi Phuong Ngoc, Nguyen Thanh Long, 
\textit{A mixed Dirichlet-Robin problem for a nonlinear Kirchhoff-Carrier
wave equation}, Nonlinear Anal. RWA. \textbf{13} (2) (2012) 817-839.
\end{thebibliography}
\end{frame}


\begin{frame}
    \begin{center}
        \Huge Cảm ơn Thầy, Cô và các bạn đã theo dõi!
    \end{center}
\end{frame}

\end{document}
